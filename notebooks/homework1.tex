
% Default to the notebook output style

    


% Inherit from the specified cell style.




    
\documentclass[11pt]{article}

    
    
    \usepackage[T1]{fontenc}
    % Nicer default font than Computer Modern for most use cases
    \usepackage{palatino}

    % Basic figure setup, for now with no caption control since it's done
    % automatically by Pandoc (which extracts ![](path) syntax from Markdown).
    \usepackage{graphicx}
    % We will generate all images so they have a width \maxwidth. This means
    % that they will get their normal width if they fit onto the page, but
    % are scaled down if they would overflow the margins.
    \makeatletter
    \def\maxwidth{\ifdim\Gin@nat@width>\linewidth\linewidth
    \else\Gin@nat@width\fi}
    \makeatother
    \let\Oldincludegraphics\includegraphics
    % Set max figure width to be 80% of text width, for now hardcoded.
    \renewcommand{\includegraphics}[1]{\Oldincludegraphics[width=.8\maxwidth]{#1}}
    % Ensure that by default, figures have no caption (until we provide a
    % proper Figure object with a Caption API and a way to capture that
    % in the conversion process - todo).
    \usepackage{caption}
    \DeclareCaptionLabelFormat{nolabel}{}
    \captionsetup{labelformat=nolabel}

    \usepackage{adjustbox} % Used to constrain images to a maximum size 
    \usepackage{xcolor} % Allow colors to be defined
    \usepackage{enumerate} % Needed for markdown enumerations to work
    \usepackage{geometry} % Used to adjust the document margins
    \usepackage{amsmath} % Equations
    \usepackage{amssymb} % Equations
    \usepackage{textcomp} % defines textquotesingle
    % Hack from http://tex.stackexchange.com/a/47451/13684:
    \AtBeginDocument{%
        \def\PYZsq{\textquotesingle}% Upright quotes in Pygmentized code
    }
    \usepackage{upquote} % Upright quotes for verbatim code
    \usepackage{eurosym} % defines \euro
    \usepackage[mathletters]{ucs} % Extended unicode (utf-8) support
    \usepackage[utf8x]{inputenc} % Allow utf-8 characters in the tex document
    \usepackage{fancyvrb} % verbatim replacement that allows latex
    \usepackage{grffile} % extends the file name processing of package graphics 
                         % to support a larger range 
    % The hyperref package gives us a pdf with properly built
    % internal navigation ('pdf bookmarks' for the table of contents,
    % internal cross-reference links, web links for URLs, etc.)
    \usepackage{hyperref}
    \usepackage{longtable} % longtable support required by pandoc >1.10
    \usepackage{booktabs}  % table support for pandoc > 1.12.2
    \usepackage[normalem]{ulem} % ulem is needed to support strikethroughs (\sout)
                                % normalem makes italics be italics, not underlines
    

    
    
    % Colors for the hyperref package
    \definecolor{urlcolor}{rgb}{0,.145,.698}
    \definecolor{linkcolor}{rgb}{.71,0.21,0.01}
    \definecolor{citecolor}{rgb}{.12,.54,.11}

    % ANSI colors
    \definecolor{ansi-black}{HTML}{3E424D}
    \definecolor{ansi-black-intense}{HTML}{282C36}
    \definecolor{ansi-red}{HTML}{E75C58}
    \definecolor{ansi-red-intense}{HTML}{B22B31}
    \definecolor{ansi-green}{HTML}{00A250}
    \definecolor{ansi-green-intense}{HTML}{007427}
    \definecolor{ansi-yellow}{HTML}{DDB62B}
    \definecolor{ansi-yellow-intense}{HTML}{B27D12}
    \definecolor{ansi-blue}{HTML}{208FFB}
    \definecolor{ansi-blue-intense}{HTML}{0065CA}
    \definecolor{ansi-magenta}{HTML}{D160C4}
    \definecolor{ansi-magenta-intense}{HTML}{A03196}
    \definecolor{ansi-cyan}{HTML}{60C6C8}
    \definecolor{ansi-cyan-intense}{HTML}{258F8F}
    \definecolor{ansi-white}{HTML}{C5C1B4}
    \definecolor{ansi-white-intense}{HTML}{A1A6B2}

    % commands and environments needed by pandoc snippets
    % extracted from the output of `pandoc -s`
    \providecommand{\tightlist}{%
      \setlength{\itemsep}{0pt}\setlength{\parskip}{0pt}}
    \DefineVerbatimEnvironment{Highlighting}{Verbatim}{commandchars=\\\{\}}
    % Add ',fontsize=\small' for more characters per line
    \newenvironment{Shaded}{}{}
    \newcommand{\KeywordTok}[1]{\textcolor[rgb]{0.00,0.44,0.13}{\textbf{{#1}}}}
    \newcommand{\DataTypeTok}[1]{\textcolor[rgb]{0.56,0.13,0.00}{{#1}}}
    \newcommand{\DecValTok}[1]{\textcolor[rgb]{0.25,0.63,0.44}{{#1}}}
    \newcommand{\BaseNTok}[1]{\textcolor[rgb]{0.25,0.63,0.44}{{#1}}}
    \newcommand{\FloatTok}[1]{\textcolor[rgb]{0.25,0.63,0.44}{{#1}}}
    \newcommand{\CharTok}[1]{\textcolor[rgb]{0.25,0.44,0.63}{{#1}}}
    \newcommand{\StringTok}[1]{\textcolor[rgb]{0.25,0.44,0.63}{{#1}}}
    \newcommand{\CommentTok}[1]{\textcolor[rgb]{0.38,0.63,0.69}{\textit{{#1}}}}
    \newcommand{\OtherTok}[1]{\textcolor[rgb]{0.00,0.44,0.13}{{#1}}}
    \newcommand{\AlertTok}[1]{\textcolor[rgb]{1.00,0.00,0.00}{\textbf{{#1}}}}
    \newcommand{\FunctionTok}[1]{\textcolor[rgb]{0.02,0.16,0.49}{{#1}}}
    \newcommand{\RegionMarkerTok}[1]{{#1}}
    \newcommand{\ErrorTok}[1]{\textcolor[rgb]{1.00,0.00,0.00}{\textbf{{#1}}}}
    \newcommand{\NormalTok}[1]{{#1}}
    
    % Additional commands for more recent versions of Pandoc
    \newcommand{\ConstantTok}[1]{\textcolor[rgb]{0.53,0.00,0.00}{{#1}}}
    \newcommand{\SpecialCharTok}[1]{\textcolor[rgb]{0.25,0.44,0.63}{{#1}}}
    \newcommand{\VerbatimStringTok}[1]{\textcolor[rgb]{0.25,0.44,0.63}{{#1}}}
    \newcommand{\SpecialStringTok}[1]{\textcolor[rgb]{0.73,0.40,0.53}{{#1}}}
    \newcommand{\ImportTok}[1]{{#1}}
    \newcommand{\DocumentationTok}[1]{\textcolor[rgb]{0.73,0.13,0.13}{\textit{{#1}}}}
    \newcommand{\AnnotationTok}[1]{\textcolor[rgb]{0.38,0.63,0.69}{\textbf{\textit{{#1}}}}}
    \newcommand{\CommentVarTok}[1]{\textcolor[rgb]{0.38,0.63,0.69}{\textbf{\textit{{#1}}}}}
    \newcommand{\VariableTok}[1]{\textcolor[rgb]{0.10,0.09,0.49}{{#1}}}
    \newcommand{\ControlFlowTok}[1]{\textcolor[rgb]{0.00,0.44,0.13}{\textbf{{#1}}}}
    \newcommand{\OperatorTok}[1]{\textcolor[rgb]{0.40,0.40,0.40}{{#1}}}
    \newcommand{\BuiltInTok}[1]{{#1}}
    \newcommand{\ExtensionTok}[1]{{#1}}
    \newcommand{\PreprocessorTok}[1]{\textcolor[rgb]{0.74,0.48,0.00}{{#1}}}
    \newcommand{\AttributeTok}[1]{\textcolor[rgb]{0.49,0.56,0.16}{{#1}}}
    \newcommand{\InformationTok}[1]{\textcolor[rgb]{0.38,0.63,0.69}{\textbf{\textit{{#1}}}}}
    \newcommand{\WarningTok}[1]{\textcolor[rgb]{0.38,0.63,0.69}{\textbf{\textit{{#1}}}}}
    
    
    % Define a nice break command that doesn't care if a line doesn't already
    % exist.
    \def\br{\hspace*{\fill} \\* }
    % Math Jax compatability definitions
    \def\gt{>}
    \def\lt{<}
    % Document parameters
    \title{homework1}
    
    
    

    % Pygments definitions
    
\makeatletter
\def\PY@reset{\let\PY@it=\relax \let\PY@bf=\relax%
    \let\PY@ul=\relax \let\PY@tc=\relax%
    \let\PY@bc=\relax \let\PY@ff=\relax}
\def\PY@tok#1{\csname PY@tok@#1\endcsname}
\def\PY@toks#1+{\ifx\relax#1\empty\else%
    \PY@tok{#1}\expandafter\PY@toks\fi}
\def\PY@do#1{\PY@bc{\PY@tc{\PY@ul{%
    \PY@it{\PY@bf{\PY@ff{#1}}}}}}}
\def\PY#1#2{\PY@reset\PY@toks#1+\relax+\PY@do{#2}}

\expandafter\def\csname PY@tok@gd\endcsname{\def\PY@tc##1{\textcolor[rgb]{0.63,0.00,0.00}{##1}}}
\expandafter\def\csname PY@tok@gu\endcsname{\let\PY@bf=\textbf\def\PY@tc##1{\textcolor[rgb]{0.50,0.00,0.50}{##1}}}
\expandafter\def\csname PY@tok@gt\endcsname{\def\PY@tc##1{\textcolor[rgb]{0.00,0.27,0.87}{##1}}}
\expandafter\def\csname PY@tok@gs\endcsname{\let\PY@bf=\textbf}
\expandafter\def\csname PY@tok@gr\endcsname{\def\PY@tc##1{\textcolor[rgb]{1.00,0.00,0.00}{##1}}}
\expandafter\def\csname PY@tok@cm\endcsname{\let\PY@it=\textit\def\PY@tc##1{\textcolor[rgb]{0.25,0.50,0.50}{##1}}}
\expandafter\def\csname PY@tok@vg\endcsname{\def\PY@tc##1{\textcolor[rgb]{0.10,0.09,0.49}{##1}}}
\expandafter\def\csname PY@tok@vi\endcsname{\def\PY@tc##1{\textcolor[rgb]{0.10,0.09,0.49}{##1}}}
\expandafter\def\csname PY@tok@mh\endcsname{\def\PY@tc##1{\textcolor[rgb]{0.40,0.40,0.40}{##1}}}
\expandafter\def\csname PY@tok@cs\endcsname{\let\PY@it=\textit\def\PY@tc##1{\textcolor[rgb]{0.25,0.50,0.50}{##1}}}
\expandafter\def\csname PY@tok@ge\endcsname{\let\PY@it=\textit}
\expandafter\def\csname PY@tok@vc\endcsname{\def\PY@tc##1{\textcolor[rgb]{0.10,0.09,0.49}{##1}}}
\expandafter\def\csname PY@tok@il\endcsname{\def\PY@tc##1{\textcolor[rgb]{0.40,0.40,0.40}{##1}}}
\expandafter\def\csname PY@tok@go\endcsname{\def\PY@tc##1{\textcolor[rgb]{0.53,0.53,0.53}{##1}}}
\expandafter\def\csname PY@tok@cp\endcsname{\def\PY@tc##1{\textcolor[rgb]{0.74,0.48,0.00}{##1}}}
\expandafter\def\csname PY@tok@gi\endcsname{\def\PY@tc##1{\textcolor[rgb]{0.00,0.63,0.00}{##1}}}
\expandafter\def\csname PY@tok@gh\endcsname{\let\PY@bf=\textbf\def\PY@tc##1{\textcolor[rgb]{0.00,0.00,0.50}{##1}}}
\expandafter\def\csname PY@tok@ni\endcsname{\let\PY@bf=\textbf\def\PY@tc##1{\textcolor[rgb]{0.60,0.60,0.60}{##1}}}
\expandafter\def\csname PY@tok@nl\endcsname{\def\PY@tc##1{\textcolor[rgb]{0.63,0.63,0.00}{##1}}}
\expandafter\def\csname PY@tok@nn\endcsname{\let\PY@bf=\textbf\def\PY@tc##1{\textcolor[rgb]{0.00,0.00,1.00}{##1}}}
\expandafter\def\csname PY@tok@no\endcsname{\def\PY@tc##1{\textcolor[rgb]{0.53,0.00,0.00}{##1}}}
\expandafter\def\csname PY@tok@na\endcsname{\def\PY@tc##1{\textcolor[rgb]{0.49,0.56,0.16}{##1}}}
\expandafter\def\csname PY@tok@nb\endcsname{\def\PY@tc##1{\textcolor[rgb]{0.00,0.50,0.00}{##1}}}
\expandafter\def\csname PY@tok@nc\endcsname{\let\PY@bf=\textbf\def\PY@tc##1{\textcolor[rgb]{0.00,0.00,1.00}{##1}}}
\expandafter\def\csname PY@tok@nd\endcsname{\def\PY@tc##1{\textcolor[rgb]{0.67,0.13,1.00}{##1}}}
\expandafter\def\csname PY@tok@ne\endcsname{\let\PY@bf=\textbf\def\PY@tc##1{\textcolor[rgb]{0.82,0.25,0.23}{##1}}}
\expandafter\def\csname PY@tok@nf\endcsname{\def\PY@tc##1{\textcolor[rgb]{0.00,0.00,1.00}{##1}}}
\expandafter\def\csname PY@tok@si\endcsname{\let\PY@bf=\textbf\def\PY@tc##1{\textcolor[rgb]{0.73,0.40,0.53}{##1}}}
\expandafter\def\csname PY@tok@s2\endcsname{\def\PY@tc##1{\textcolor[rgb]{0.73,0.13,0.13}{##1}}}
\expandafter\def\csname PY@tok@nt\endcsname{\let\PY@bf=\textbf\def\PY@tc##1{\textcolor[rgb]{0.00,0.50,0.00}{##1}}}
\expandafter\def\csname PY@tok@nv\endcsname{\def\PY@tc##1{\textcolor[rgb]{0.10,0.09,0.49}{##1}}}
\expandafter\def\csname PY@tok@s1\endcsname{\def\PY@tc##1{\textcolor[rgb]{0.73,0.13,0.13}{##1}}}
\expandafter\def\csname PY@tok@ch\endcsname{\let\PY@it=\textit\def\PY@tc##1{\textcolor[rgb]{0.25,0.50,0.50}{##1}}}
\expandafter\def\csname PY@tok@m\endcsname{\def\PY@tc##1{\textcolor[rgb]{0.40,0.40,0.40}{##1}}}
\expandafter\def\csname PY@tok@gp\endcsname{\let\PY@bf=\textbf\def\PY@tc##1{\textcolor[rgb]{0.00,0.00,0.50}{##1}}}
\expandafter\def\csname PY@tok@sh\endcsname{\def\PY@tc##1{\textcolor[rgb]{0.73,0.13,0.13}{##1}}}
\expandafter\def\csname PY@tok@ow\endcsname{\let\PY@bf=\textbf\def\PY@tc##1{\textcolor[rgb]{0.67,0.13,1.00}{##1}}}
\expandafter\def\csname PY@tok@sx\endcsname{\def\PY@tc##1{\textcolor[rgb]{0.00,0.50,0.00}{##1}}}
\expandafter\def\csname PY@tok@bp\endcsname{\def\PY@tc##1{\textcolor[rgb]{0.00,0.50,0.00}{##1}}}
\expandafter\def\csname PY@tok@c1\endcsname{\let\PY@it=\textit\def\PY@tc##1{\textcolor[rgb]{0.25,0.50,0.50}{##1}}}
\expandafter\def\csname PY@tok@o\endcsname{\def\PY@tc##1{\textcolor[rgb]{0.40,0.40,0.40}{##1}}}
\expandafter\def\csname PY@tok@kc\endcsname{\let\PY@bf=\textbf\def\PY@tc##1{\textcolor[rgb]{0.00,0.50,0.00}{##1}}}
\expandafter\def\csname PY@tok@c\endcsname{\let\PY@it=\textit\def\PY@tc##1{\textcolor[rgb]{0.25,0.50,0.50}{##1}}}
\expandafter\def\csname PY@tok@mf\endcsname{\def\PY@tc##1{\textcolor[rgb]{0.40,0.40,0.40}{##1}}}
\expandafter\def\csname PY@tok@err\endcsname{\def\PY@bc##1{\setlength{\fboxsep}{0pt}\fcolorbox[rgb]{1.00,0.00,0.00}{1,1,1}{\strut ##1}}}
\expandafter\def\csname PY@tok@mb\endcsname{\def\PY@tc##1{\textcolor[rgb]{0.40,0.40,0.40}{##1}}}
\expandafter\def\csname PY@tok@ss\endcsname{\def\PY@tc##1{\textcolor[rgb]{0.10,0.09,0.49}{##1}}}
\expandafter\def\csname PY@tok@sr\endcsname{\def\PY@tc##1{\textcolor[rgb]{0.73,0.40,0.53}{##1}}}
\expandafter\def\csname PY@tok@mo\endcsname{\def\PY@tc##1{\textcolor[rgb]{0.40,0.40,0.40}{##1}}}
\expandafter\def\csname PY@tok@kd\endcsname{\let\PY@bf=\textbf\def\PY@tc##1{\textcolor[rgb]{0.00,0.50,0.00}{##1}}}
\expandafter\def\csname PY@tok@mi\endcsname{\def\PY@tc##1{\textcolor[rgb]{0.40,0.40,0.40}{##1}}}
\expandafter\def\csname PY@tok@kn\endcsname{\let\PY@bf=\textbf\def\PY@tc##1{\textcolor[rgb]{0.00,0.50,0.00}{##1}}}
\expandafter\def\csname PY@tok@cpf\endcsname{\let\PY@it=\textit\def\PY@tc##1{\textcolor[rgb]{0.25,0.50,0.50}{##1}}}
\expandafter\def\csname PY@tok@kr\endcsname{\let\PY@bf=\textbf\def\PY@tc##1{\textcolor[rgb]{0.00,0.50,0.00}{##1}}}
\expandafter\def\csname PY@tok@s\endcsname{\def\PY@tc##1{\textcolor[rgb]{0.73,0.13,0.13}{##1}}}
\expandafter\def\csname PY@tok@kp\endcsname{\def\PY@tc##1{\textcolor[rgb]{0.00,0.50,0.00}{##1}}}
\expandafter\def\csname PY@tok@w\endcsname{\def\PY@tc##1{\textcolor[rgb]{0.73,0.73,0.73}{##1}}}
\expandafter\def\csname PY@tok@kt\endcsname{\def\PY@tc##1{\textcolor[rgb]{0.69,0.00,0.25}{##1}}}
\expandafter\def\csname PY@tok@sc\endcsname{\def\PY@tc##1{\textcolor[rgb]{0.73,0.13,0.13}{##1}}}
\expandafter\def\csname PY@tok@sb\endcsname{\def\PY@tc##1{\textcolor[rgb]{0.73,0.13,0.13}{##1}}}
\expandafter\def\csname PY@tok@k\endcsname{\let\PY@bf=\textbf\def\PY@tc##1{\textcolor[rgb]{0.00,0.50,0.00}{##1}}}
\expandafter\def\csname PY@tok@se\endcsname{\let\PY@bf=\textbf\def\PY@tc##1{\textcolor[rgb]{0.73,0.40,0.13}{##1}}}
\expandafter\def\csname PY@tok@sd\endcsname{\let\PY@it=\textit\def\PY@tc##1{\textcolor[rgb]{0.73,0.13,0.13}{##1}}}

\def\PYZbs{\char`\\}
\def\PYZus{\char`\_}
\def\PYZob{\char`\{}
\def\PYZcb{\char`\}}
\def\PYZca{\char`\^}
\def\PYZam{\char`\&}
\def\PYZlt{\char`\<}
\def\PYZgt{\char`\>}
\def\PYZsh{\char`\#}
\def\PYZpc{\char`\%}
\def\PYZdl{\char`\$}
\def\PYZhy{\char`\-}
\def\PYZsq{\char`\'}
\def\PYZdq{\char`\"}
\def\PYZti{\char`\~}
% for compatibility with earlier versions
\def\PYZat{@}
\def\PYZlb{[}
\def\PYZrb{]}
\makeatother


    % Exact colors from NB
    \definecolor{incolor}{rgb}{0.0, 0.0, 0.5}
    \definecolor{outcolor}{rgb}{0.545, 0.0, 0.0}



    
    % Prevent overflowing lines due to hard-to-break entities
    \sloppy 
    % Setup hyperref package
    \hypersetup{
      breaklinks=true,  % so long urls are correctly broken across lines
      colorlinks=true,
      urlcolor=urlcolor,
      linkcolor=linkcolor,
      citecolor=citecolor,
      }
    % Slightly bigger margins than the latex defaults
    
    \geometry{verbose,tmargin=1in,bmargin=1in,lmargin=1in,rmargin=1in}
    
    

    \begin{document}
    
    
    \maketitle
    
    

    
    \textbf{Jacob Lustig-Yaeger}\\
\textbf{ASTR 555 - Planetary Atmospheres}\\
\textbf{Prof.~David Catling}

\emph{Preamble--} The following homework submission is the output file
from an iPython Notebook and contains the original code written and run
for the calculations in this assignment. Including the code does make
this a lengthy document (perhaps unnecessarily lengthy), but I think
it's a pretty nifty way to provide my answers and show my work.

    \begin{Verbatim}[commandchars=\\\{\}]
{\color{incolor}In [{\color{incolor}56}]:} \PY{c+c1}{\PYZsh{} Import numpy for math}
         \PY{k+kn}{import} \PY{n+nn}{numpy} \PY{k+kn}{as} \PY{n+nn}{np}
         
         \PY{c+c1}{\PYZsh{} Import units to keep me honest}
         \PY{k+kn}{import} \PY{n+nn}{astropy.units} \PY{k+kn}{as} \PY{n+nn}{u}
\end{Verbatim}

    \subsection{Problem 1:}\label{problem-1}

\begin{quote}
{[}16 points total{]} In the future, planetary astronomers discover an
inhabited extrasolar planet, which they call ``Houston2''. This planet
has a foul and noxious polluted atmosphere, which the media speculates
is caused by incessant, unregulated industrial activity. However, all
that is really known is that the atmosphere is rich in hydrocarbons,
that the temperature at the 200-mbar level in the atmosphere is 273.16
K, and that the surface temperature is 546.32 K. The astronomers all
agree that the atmosphere has an essentially adiabatic structure. But
there are three camps regarding the atmospheric composition: (i) one
group believes that the atmosphere is 50\% by volume Xe and 50\%
methane. (ii) the other group believes that the atmosphere is pure lead
tetraethyl Pb(C2H5)4 --- the unwelcome substance found in leaded
gasoline (iii) a final group argue for a composition of 50\% He and 50\%
methane.
\end{quote}

\begin{quote}
\begin{enumerate}
\def\labelenumi{(\alph{enumi})}
\tightlist
\item
  What is the surface atmospheric pressure on ``Houston2'' in each model
  (i) through (iii)? {[}12 pts{]}\\
\item
  Explain how atmospheres (i) and (iii) will differ in their vertical
  extent? You can assume that Houston2 has the same gravity as the
  Earth, and a useful thing to do would be to compare heights at the 0.2
  bar level. {[}4 pts{]}
\end{enumerate}
\end{quote}

\begin{quote}
{[}In answering part (a), you will need heat capacity. Go and look at a
thermal physics textbook. You'll find that each degree of freedom of a
molecule or atom contributes R/2 to Cv, the molar heat capacity at
constant volume. For a molecule of N atoms, there are 3N degrees of
freedom. For example, for He there are 3N = 3 degrees of freedom
corresponding to 3 translational modes in the x, y and z directions, so
that Cv = 3R/2. For CO, for example, there are 3N = 6 degrees of freedom
corresponding to 3 translational modes, 1 vibrational mode and 2
rotational modes. But whether certain modes are excited depends on
temperature. H2, for example, has Cv = 3R/2 at 40K, Cv \textasciitilde{}
5R/2 at 300K and CV = 6R/2 at 2000K and approaches 7R/2 in the limit (H2
dissociates before it gets there). Each vibrational mode has a kinetic
part associated with movement of atoms and a potential part like spring
energy in a simple harmonic oscillator, which can constitute two dofs.
So the maximum possible Cv = (3 trans. + 2 rots + 2 vibs)R/2 = 7R/2 for
a diatomic molecule. At room temperature, vibrational modes tend to be
inactive, so Cv = 5R/2 for most diatomic molecules. For the purpose of
this homework, ignore these complications and assume that the 3N rule
holds. Recall also that molar heat capacity at constant pressure is
related to Cv via the equation Cp = Cv + R. (Data: Masses in grams/mole:
Xe = 130, C = 12, H = 1, He = 4, Pb = 206){]}.
\end{quote}

Let's start by cataloging the given quantities:

    \begin{Verbatim}[commandchars=\\\{\}]
{\color{incolor}In [{\color{incolor}61}]:} \PY{n}{T200} \PY{o}{=} \PY{l+m+mf}{273.16} \PY{o}{*} \PY{n}{u}\PY{o}{.}\PY{n}{K}\PY{p}{;} \PY{k}{print} \PY{n}{T200}
         \PY{n}{P200} \PY{o}{=} \PY{l+m+mf}{200.0} \PY{o}{*} \PY{l+m+mf}{1e\PYZhy{}3} \PY{o}{*} \PY{n}{u}\PY{o}{.}\PY{n}{bar}\PY{p}{;} \PY{k}{print} \PY{n}{P200}
         \PY{n}{Tsurf} \PY{o}{=} \PY{l+m+mf}{546.32} \PY{o}{*} \PY{n}{u}\PY{o}{.}\PY{n}{K}\PY{p}{;} \PY{k}{print} \PY{n}{Tsurf}
\end{Verbatim}

    \begin{Verbatim}[commandchars=\\\{\}]
273.16 K
0.2 bar
546.32 K

    \end{Verbatim}

    Define gram/mol unit

    \begin{Verbatim}[commandchars=\\\{\}]
{\color{incolor}In [{\color{incolor}59}]:} \PY{n}{amu} \PY{o}{=} \PY{n}{u}\PY{o}{.}\PY{n}{gram} \PY{o}{/} \PY{n}{u}\PY{o}{.}\PY{n}{mol}\PY{p}{;} \PY{n}{amu}
\end{Verbatim}
\texttt{\color{outcolor}Out[{\color{outcolor}59}]:}
    
    \(\mathrm{\frac{g}{mol}}\)

    

    The masses of atoms in grams/mol are:

    \begin{Verbatim}[commandchars=\\\{\}]
{\color{incolor}In [{\color{incolor} }]:} \PY{n}{mmw\PYZus{}Xe} \PY{o}{=} \PY{l+m+mf}{130.} \PY{o}{*} \PY{n}{amu}
        \PY{n}{mmw\PYZus{}C} \PY{o}{=} \PY{l+m+mf}{12.} \PY{o}{*} \PY{n}{amu}
        \PY{n}{mmw\PYZus{}H} \PY{o}{=} \PY{l+m+mf}{1.0} \PY{o}{*}\PY{n}{amu}
        \PY{n}{mmw\PYZus{}He} \PY{o}{=} \PY{l+m+mf}{4.0} \PY{o}{*} \PY{n}{amu}
        \PY{n}{mmw\PYZus{}Pb} \PY{o}{=} \PY{l+m+mf}{206.} \PY{o}{*} \PY{n}{amu}
\end{Verbatim}

    The masses of the molecules are:

    \begin{Verbatim}[commandchars=\\\{\}]
{\color{incolor}In [{\color{incolor}63}]:} \PY{c+c1}{\PYZsh{} Mass of methane}
         \PY{n}{mmw\PYZus{}CH4} \PY{o}{=} \PY{n}{mmw\PYZus{}C} \PY{o}{+} \PY{l+m+mi}{4}\PY{o}{*}\PY{n}{mmw\PYZus{}H}\PY{p}{;} \PY{n}{mmw\PYZus{}CH4}
\end{Verbatim}
\texttt{\color{outcolor}Out[{\color{outcolor}63}]:}
    
    \(16 \; \mathrm{\frac{g}{mol}}\)

    

    \begin{Verbatim}[commandchars=\\\{\}]
{\color{incolor}In [{\color{incolor}64}]:} \PY{c+c1}{\PYZsh{} Mass of Lead Tetraethyl}
         \PY{n}{mmw\PYZus{}PbTet} \PY{o}{=} \PY{n}{mmw\PYZus{}Pb} \PY{o}{+} \PY{l+m+mi}{4}\PY{o}{*}\PY{p}{(}\PY{l+m+mi}{2}\PY{o}{*}\PY{n}{mmw\PYZus{}C}\PY{o}{+}\PY{l+m+mi}{5}\PY{o}{*}\PY{n}{mmw\PYZus{}H}\PY{p}{)}\PY{p}{;} \PY{n}{mmw\PYZus{}PbTet}
\end{Verbatim}
\texttt{\color{outcolor}Out[{\color{outcolor}64}]:}
    
    \(322 \; \mathrm{\frac{g}{mol}}\)

    

    Now each case has a different composition, given in terms of volume
mixing ratio.

    \begin{Verbatim}[commandchars=\\\{\}]
{\color{incolor}In [{\color{incolor}66}]:} \PY{c+c1}{\PYZsh{} Cases}
         \PY{c+c1}{\PYZsh{}(i)}
         \PY{n}{fXe} \PY{o}{=} \PY{l+m+mf}{0.5}
         \PY{n}{fCH4} \PY{o}{=} \PY{l+m+mf}{0.5}
         \PY{c+c1}{\PYZsh{}(ii)}
         \PY{n}{fPbTet} \PY{o}{=} \PY{l+m+mf}{1.0}
         \PY{c+c1}{\PYZsh{}(iii)}
         \PY{n}{fHe} \PY{o}{=} \PY{l+m+mf}{0.5}
         \PY{n}{fCH4} \PY{o}{=} \PY{l+m+mf}{0.5}
\end{Verbatim}

    \subsubsection{Part a)}\label{part-a}

From the notes,\\
\[ T = T_{ref} \left ( \frac{P}{P_{ref}} \right ) ^{\frac{\gamma - 1}{\gamma}} \]\\
where \(\gamma\) is the ratio of specific heats (\(\gamma = c_p/c_v\)).
Solving for P,\\
\[ P = P_{ref} \left ( \frac{T}{T_{ref}} \right ) ^{\frac{\gamma}{\gamma - 1}} \]
which can be used to find the surface pressure using the P,T known at
200 mbar and the surface temperature. We can use kinetic theory to
relate \(\gamma\) to the number of degrees of freedom available to a gas
particle via the following relation,\\
\[ \gamma = 1 + \frac{2}{N_{dof}} .\] Finally, using the simplifying
assumption that \(N_{dof} = 3N\), where \(N\) is the number of atoms, we
get the following relation:\\
\[ P = P_{ref} \left ( \frac{T}{T_{ref}} \right ) ^{1+\frac{N_{dof}}{2}} \]

Let's write the above equation as a function so we can call it multiple
times.

    \begin{Verbatim}[commandchars=\\\{\}]
{\color{incolor}In [{\color{incolor}65}]:} \PY{k}{def} \PY{n+nf}{P1}\PY{p}{(}\PY{n}{T}\PY{p}{,} \PY{n}{Pr}\PY{p}{,} \PY{n}{Tr}\PY{p}{,} \PY{n}{Ndof}\PY{p}{)}\PY{p}{:}
             \PY{l+s+sd}{\PYZdq{}\PYZdq{}\PYZdq{}}
         \PY{l+s+sd}{    Calculate the pressure assuming a known reference temperature and pressure, a deeper temperature, }
         \PY{l+s+sd}{    and assuming a dry adiabat.}
         \PY{l+s+sd}{    }
         \PY{l+s+sd}{    Parameters}
         \PY{l+s+sd}{    \PYZhy{}\PYZhy{}\PYZhy{}\PYZhy{}\PYZhy{}\PYZhy{}\PYZhy{}\PYZhy{}\PYZhy{}\PYZhy{}}
         \PY{l+s+sd}{    T : Temperature at which to determine the pressure}
         \PY{l+s+sd}{    Pr : Reference pressure}
         \PY{l+s+sd}{    Tr : Reference temperature}
         \PY{l+s+sd}{    Ndof : Gas number of degrees of freedom}
         \PY{l+s+sd}{    }
         \PY{l+s+sd}{    Returns}
         \PY{l+s+sd}{    \PYZhy{}\PYZhy{}\PYZhy{}\PYZhy{}\PYZhy{}\PYZhy{}\PYZhy{}}
         \PY{l+s+sd}{    P1 : Pressure at given T}
         \PY{l+s+sd}{    \PYZdq{}\PYZdq{}\PYZdq{}}
             \PY{k}{return} \PY{n}{Pr} \PY{o}{*} \PY{p}{(}\PY{n}{T} \PY{o}{/} \PY{n}{Tr}\PY{p}{)}\PY{o}{*}\PY{o}{*}\PY{p}{(}\PY{l+m+mf}{1.}\PY{o}{+}\PY{n}{Ndof}\PY{o}{/}\PY{l+m+mf}{2.}\PY{p}{)}
\end{Verbatim}

    We will determine the number of degrees of freedom in a mixed gas by
weighting the individual gas constituents by their respective volume
mixing ratios.

    \begin{Verbatim}[commandchars=\\\{\}]
{\color{incolor}In [{\color{incolor}67}]:} \PY{c+c1}{\PYZsh{} Set the number of atoms for each gas}
         \PY{n}{N\PYZus{}Xe} \PY{o}{=} \PY{l+m+mf}{1.}
         \PY{n}{N\PYZus{}CH4} \PY{o}{=} \PY{l+m+mf}{5.}
         \PY{n}{N\PYZus{}PbT} \PY{o}{=} \PY{l+m+mf}{29.}
         \PY{n}{N\PYZus{}He} \PY{o}{=} \PY{l+m+mf}{1.0}
         
         \PY{c+c1}{\PYZsh{} Compute the Number of degrees of freedom for each case}
         \PY{n}{Ndof1} \PY{o}{=} \PY{l+m+mf}{3.} \PY{o}{*} \PY{p}{(}\PY{n}{fXe}\PY{o}{*}\PY{n}{N\PYZus{}Xe} \PY{o}{+} \PY{n}{fCH4}\PY{o}{*}\PY{n}{N\PYZus{}CH4}\PY{p}{)}
         \PY{n}{Ndof2} \PY{o}{=} \PY{l+m+mf}{3.} \PY{o}{*} \PY{p}{(}\PY{n}{fPbTet}\PY{o}{*}\PY{n}{N\PYZus{}PbT}\PY{p}{)}
         \PY{n}{Ndof3} \PY{o}{=} \PY{l+m+mf}{3.} \PY{o}{*} \PY{p}{(}\PY{n}{fHe}\PY{o}{*}\PY{n}{N\PYZus{}He} \PY{o}{+} \PY{n}{fCH4}\PY{o}{*}\PY{n}{N\PYZus{}CH4}\PY{p}{)}
         
         \PY{c+c1}{\PYZsh{} print intermediate values for checking math}
         \PY{k}{print} \PY{l+s+s2}{\PYZdq{}}\PY{l+s+s2}{N\PYZus{}dof:}\PY{l+s+s2}{\PYZdq{}}\PY{p}{,} \PY{n}{Ndof1}\PY{p}{,} \PY{n}{Ndof2}\PY{p}{,} \PY{n}{Ndof3}
         \PY{k}{print} \PY{l+s+s2}{\PYZdq{}}\PY{l+s+s2}{gamma:}\PY{l+s+s2}{\PYZdq{}}\PY{p}{,} \PY{p}{(}\PY{l+m+mf}{1.}\PY{o}{+}\PY{l+m+mf}{2.}\PY{o}{/}\PY{n}{Ndof1}\PY{p}{)}\PY{p}{,} \PY{p}{(}\PY{l+m+mf}{1.}\PY{o}{+}\PY{l+m+mf}{2.}\PY{o}{/}\PY{n}{Ndof2}\PY{p}{)}\PY{p}{,} \PY{p}{(}\PY{l+m+mf}{1.}\PY{o}{+}\PY{l+m+mf}{2.}\PY{o}{/}\PY{n}{Ndof3}\PY{p}{)}
         \PY{k}{print} \PY{l+s+s2}{\PYZdq{}}\PY{l+s+s2}{exponent:}\PY{l+s+s2}{\PYZdq{}}\PY{p}{,} \PY{l+m+mi}{1}  \PY{o}{+} \PY{n}{Ndof1}\PY{o}{/}\PY{l+m+mf}{2.}\PY{p}{,} \PY{l+m+mi}{1}  \PY{o}{+} \PY{n}{Ndof2}\PY{o}{/}\PY{l+m+mf}{2.}\PY{p}{,} \PY{l+m+mi}{1}  \PY{o}{+} \PY{n}{Ndof3}\PY{o}{/}\PY{l+m+mf}{2.}
\end{Verbatim}

    \begin{Verbatim}[commandchars=\\\{\}]
N\_dof: 9.0 87.0 9.0
gamma: 1.22222222222 1.02298850575 1.22222222222
exponent: 5.5 44.5 5.5

    \end{Verbatim}

    Given a different number of degrees of freedom for each possible
atmosphere, we calculate the surface pressures given below:

    \begin{Verbatim}[commandchars=\\\{\}]
{\color{incolor}In [{\color{incolor}7}]:} \PY{n}{Pi} \PY{o}{=} \PY{n}{P1}\PY{p}{(}\PY{n}{Tsurf}\PY{p}{,} \PY{n}{P200}\PY{p}{,} \PY{n}{T200}\PY{p}{,} \PY{n}{Ndof1}\PY{p}{)}
        \PY{k}{print} \PY{l+s+s2}{\PYZdq{}}\PY{l+s+s2}{Surface pressure for case (i) is}\PY{l+s+s2}{\PYZdq{}}\PY{p}{,} \PY{n}{Pi}
\end{Verbatim}

    \begin{Verbatim}[commandchars=\\\{\}]
Surface pressure for case (i) is 9.05096679919 bar

    \end{Verbatim}

    \begin{Verbatim}[commandchars=\\\{\}]
{\color{incolor}In [{\color{incolor}8}]:} \PY{n}{Pii} \PY{o}{=} \PY{n}{P1}\PY{p}{(}\PY{n}{Tsurf}\PY{p}{,} \PY{n}{P200}\PY{p}{,} \PY{n}{T200}\PY{p}{,} \PY{n}{Ndof2}\PY{p}{)}
        \PY{k}{print} \PY{l+s+s2}{\PYZdq{}}\PY{l+s+s2}{Surface pressure for case (ii) is}\PY{l+s+s2}{\PYZdq{}}\PY{p}{,} \PY{n}{Pii}
\end{Verbatim}

    \begin{Verbatim}[commandchars=\\\{\}]
Surface pressure for case (ii) is 4.97582161916e+12 bar

    \end{Verbatim}

    \begin{Verbatim}[commandchars=\\\{\}]
{\color{incolor}In [{\color{incolor}9}]:} \PY{n}{Piii} \PY{o}{=} \PY{n}{P1}\PY{p}{(}\PY{n}{Tsurf}\PY{p}{,} \PY{n}{P200}\PY{p}{,} \PY{n}{T200}\PY{p}{,} \PY{n}{Ndof3}\PY{p}{)}
        \PY{k}{print} \PY{l+s+s2}{\PYZdq{}}\PY{l+s+s2}{Surface pressure for case (iii) is}\PY{l+s+s2}{\PYZdq{}}\PY{p}{,} \PY{n}{Piii}
\end{Verbatim}

    \begin{Verbatim}[commandchars=\\\{\}]
Surface pressure for case (iii) is 9.05096679919 bar

    \end{Verbatim}

    \subsubsection{Part b)}\label{part-b}

The atmospheric pressure scale height is\\
\[ H = \frac{k \bar{T}}{\bar{m} g } \]

    Define the Boltzmann Constant,

    \begin{Verbatim}[commandchars=\\\{\}]
{\color{incolor}In [{\color{incolor}10}]:} \PY{n}{k} \PY{o}{=} \PY{l+m+mf}{1.38064852e\PYZhy{}23} \PY{o}{*} \PY{p}{(}\PY{n}{u}\PY{o}{.}\PY{n}{m}\PY{o}{*}\PY{o}{*}\PY{l+m+mi}{2} \PY{o}{*}\PY{n}{u}\PY{o}{.}\PY{n}{kg} \PY{o}{*} \PY{n}{u}\PY{o}{.}\PY{n}{second}\PY{o}{*}\PY{o}{*}\PY{o}{\PYZhy{}}\PY{l+m+mi}{2} \PY{o}{*} \PY{n}{u}\PY{o}{.}\PY{n}{K}\PY{o}{*}\PY{o}{*}\PY{o}{\PYZhy{}}\PY{l+m+mi}{1} \PY{o}{*} \PY{n}{u}\PY{o}{.}\PY{n}{kg}\PY{o}{.}\PY{n}{in\PYZus{}units}\PY{p}{(}\PY{n}{u}\PY{o}{.}\PY{n}{g}\PY{p}{)} \PY{o}{*} \PY{n}{u}\PY{o}{.}\PY{n}{g}\PY{o}{/}\PY{n}{u}\PY{o}{.}\PY{n}{kg}\PY{p}{)}\PY{p}{;} \PY{n}{k}
\end{Verbatim}
\texttt{\color{outcolor}Out[{\color{outcolor}10}]:}
    
    \(1.3806485 \times 10^{-20} \; \mathrm{\frac{m^{2}\,g}{K\,s^{2}}}\)

    

    and the pressure scale height.

    \begin{Verbatim}[commandchars=\\\{\}]
{\color{incolor}In [{\color{incolor}11}]:} \PY{k}{def} \PY{n+nf}{pressure\PYZus{}scale\PYZus{}height}\PY{p}{(}\PY{n}{T}\PY{p}{,} \PY{n}{m}\PY{p}{,} \PY{n}{g}\PY{p}{,} \PY{n}{k}\PY{o}{=}\PY{n}{k}\PY{p}{)}\PY{p}{:}
             \PY{l+s+sd}{\PYZdq{}\PYZdq{}\PYZdq{}}
         \PY{l+s+sd}{    \PYZdq{}\PYZdq{}\PYZdq{}}
             \PY{k}{return} \PY{p}{(}\PY{n}{k} \PY{o}{*} \PY{n}{T}\PY{p}{)} \PY{o}{/} \PY{p}{(}\PY{n}{m} \PY{o}{*} \PY{n}{g}\PY{p}{)}
\end{Verbatim}

    Assuming the gravity is the same as that on Earth

    \begin{Verbatim}[commandchars=\\\{\}]
{\color{incolor}In [{\color{incolor}12}]:} \PY{n}{g} \PY{o}{=} \PY{l+m+mf}{9.8} \PY{o}{*} \PY{n}{u}\PY{o}{.}\PY{n}{m} \PY{o}{/} \PY{n}{u}\PY{o}{.}\PY{n}{s}\PY{o}{*}\PY{o}{*}\PY{l+m+mi}{2}\PY{p}{;} \PY{n}{g}
\end{Verbatim}
\texttt{\color{outcolor}Out[{\color{outcolor}12}]:}
    
    \(9.8 \; \mathrm{\frac{m}{s^{2}}}\)

    

    Set Avagadro's Number:

    \begin{Verbatim}[commandchars=\\\{\}]
{\color{incolor}In [{\color{incolor}13}]:} \PY{n}{NA} \PY{o}{=} \PY{l+m+mf}{6.022140857e23} \PY{o}{*} \PY{n}{u}\PY{o}{.}\PY{n}{mol}\PY{o}{*}\PY{o}{*}\PY{o}{\PYZhy{}}\PY{l+m+mi}{1}\PY{p}{;} \PY{n}{NA}
\end{Verbatim}
\texttt{\color{outcolor}Out[{\color{outcolor}13}]:}
    
    \(6.0221409 \times 10^{23} \; \mathrm{\frac{1}{mol}}\)

    

    Calculate the mean molecular weight for each possible case:

    \begin{Verbatim}[commandchars=\\\{\}]
{\color{incolor}In [{\color{incolor}14}]:} \PY{n}{mu1} \PY{o}{=} \PY{p}{(}\PY{n}{fXe} \PY{o}{*} \PY{n}{mmw\PYZus{}Xe} \PY{o}{+} \PY{n}{fCH4} \PY{o}{*} \PY{n}{mmw\PYZus{}CH4}\PY{p}{)}\PY{p}{;} \PY{n}{mu1}
\end{Verbatim}
\texttt{\color{outcolor}Out[{\color{outcolor}14}]:}
    
    \(73 \; \mathrm{\frac{g}{mol}}\)

    

    \begin{Verbatim}[commandchars=\\\{\}]
{\color{incolor}In [{\color{incolor}15}]:} \PY{n}{mu2} \PY{o}{=} \PY{p}{(}\PY{n}{fPbTet} \PY{o}{*} \PY{n}{mmw\PYZus{}PbTet}\PY{p}{)}\PY{p}{;} \PY{n}{mu2}
\end{Verbatim}
\texttt{\color{outcolor}Out[{\color{outcolor}15}]:}
    
    \(322 \; \mathrm{\frac{g}{mol}}\)

    

    \begin{Verbatim}[commandchars=\\\{\}]
{\color{incolor}In [{\color{incolor}16}]:} \PY{n}{mu3} \PY{o}{=} \PY{p}{(}\PY{n}{fHe} \PY{o}{*} \PY{n}{mmw\PYZus{}He} \PY{o}{+} \PY{n}{fCH4} \PY{o}{*} \PY{n}{mmw\PYZus{}CH4}\PY{p}{)}\PY{p}{;} \PY{n}{mu3}
\end{Verbatim}
\texttt{\color{outcolor}Out[{\color{outcolor}16}]:}
    
    \(10 \; \mathrm{\frac{g}{mol}}\)

    

    Calculate the mean atmospheric mass for each possible case:

    \begin{Verbatim}[commandchars=\\\{\}]
{\color{incolor}In [{\color{incolor}17}]:} \PY{n}{m1} \PY{o}{=} \PY{n}{mu1} \PY{o}{/} \PY{n}{NA}\PY{p}{;} \PY{n}{m1}
\end{Verbatim}
\texttt{\color{outcolor}Out[{\color{outcolor}17}]:}
    
    \(1.2121935 \times 10^{-22} \; \mathrm{g}\)

    

    \begin{Verbatim}[commandchars=\\\{\}]
{\color{incolor}In [{\color{incolor}18}]:} \PY{n}{m2} \PY{o}{=} \PY{n}{mu2} \PY{o}{/} \PY{n}{NA}\PY{p}{;} \PY{n}{m2}
\end{Verbatim}
\texttt{\color{outcolor}Out[{\color{outcolor}18}]:}
    
    \(5.3469357 \times 10^{-22} \; \mathrm{g}\)

    

    \begin{Verbatim}[commandchars=\\\{\}]
{\color{incolor}In [{\color{incolor}19}]:} \PY{n}{m3} \PY{o}{=} \PY{n}{mu3} \PY{o}{/} \PY{n}{NA}\PY{p}{;} \PY{n}{m3}
\end{Verbatim}
\texttt{\color{outcolor}Out[{\color{outcolor}19}]:}
    
    \(1.660539 \times 10^{-23} \; \mathrm{g}\)

    

    We'll calculate the pressure scale height at the 200 mbar level. The
pressure scale height is only valid in isothermal regions of an
atmosphere. However, 200 mbar is probably close to the tropopause
(particularly because the temperature here is at zero degrees celsius
perhaps hinting that water will freeze out around this level,
contributing latent heat and cold trapping the water below). Using the
temperature at this level to calculate the pressure scale height will
give a reasonable approximation of the extent of the atmosphere
\emph{above} 200 mbar.

    \begin{Verbatim}[commandchars=\\\{\}]
{\color{incolor}In [{\color{incolor}20}]:} \PY{n}{H1} \PY{o}{=} \PY{n}{pressure\PYZus{}scale\PYZus{}height}\PY{p}{(}\PY{n}{T200}\PY{p}{,} \PY{n}{m1}\PY{p}{,} \PY{n}{g}\PY{p}{)} \PY{o}{*} \PY{n}{u}\PY{o}{.}\PY{n}{m}\PY{o}{.}\PY{n}{in\PYZus{}units}\PY{p}{(}\PY{n}{u}\PY{o}{.}\PY{n}{km}\PY{p}{)} \PY{o}{*} \PY{n}{u}\PY{o}{.}\PY{n}{km}\PY{o}{/}\PY{n}{u}\PY{o}{.}\PY{n}{m}\PY{p}{;} \PY{n}{H1}
\end{Verbatim}
\texttt{\color{outcolor}Out[{\color{outcolor}20}]:}
    
    \(3.1746965 \; \mathrm{km}\)

    

    \begin{Verbatim}[commandchars=\\\{\}]
{\color{incolor}In [{\color{incolor}21}]:} \PY{n}{H2} \PY{o}{=} \PY{n}{pressure\PYZus{}scale\PYZus{}height}\PY{p}{(}\PY{n}{T200}\PY{p}{,} \PY{n}{m2}\PY{p}{,} \PY{n}{g}\PY{p}{)} \PY{o}{*} \PY{n}{u}\PY{o}{.}\PY{n}{m}\PY{o}{.}\PY{n}{in\PYZus{}units}\PY{p}{(}\PY{n}{u}\PY{o}{.}\PY{n}{km}\PY{p}{)} \PY{o}{*} \PY{n}{u}\PY{o}{.}\PY{n}{km}\PY{o}{/}\PY{n}{u}\PY{o}{.}\PY{n}{m}\PY{p}{;} \PY{n}{H2}
\end{Verbatim}
\texttt{\color{outcolor}Out[{\color{outcolor}21}]:}
    
    \(0.71972932 \; \mathrm{km}\)

    

    \begin{Verbatim}[commandchars=\\\{\}]
{\color{incolor}In [{\color{incolor}22}]:} \PY{n}{H3} \PY{o}{=} \PY{n}{pressure\PYZus{}scale\PYZus{}height}\PY{p}{(}\PY{n}{T200}\PY{p}{,} \PY{n}{m3}\PY{p}{,} \PY{n}{g}\PY{p}{)} \PY{o}{*} \PY{n}{u}\PY{o}{.}\PY{n}{m}\PY{o}{.}\PY{n}{in\PYZus{}units}\PY{p}{(}\PY{n}{u}\PY{o}{.}\PY{n}{km}\PY{p}{)} \PY{o}{*} \PY{n}{u}\PY{o}{.}\PY{n}{km}\PY{o}{/}\PY{n}{u}\PY{o}{.}\PY{n}{m}\PY{p}{;} \PY{n}{H3}
\end{Verbatim}
\texttt{\color{outcolor}Out[{\color{outcolor}22}]:}
    
    \(23.175284 \; \mathrm{km}\)

    

    The pressure scale height gives us an estimate of the atmospheric
extent. Since we know the temperature at 200 mbar and the gravity, then
the scale height for the 3 possible cases is determined soley by the
atmospheric mass. The pure lead tetraethyl atmosphere, which has a mean
molecular weight of 322 g/mol, is the heaviest and therefore has the
smallest atmospheric extent with a scale height of 0.72 km. On the other
hand, the 50\% Helium, 50\% methane atmosphere has the lowest mass
(\(\mu = 10\) g/mol) and therefore the largest atmospheric extent
(\(H = 23.17\) km). The 50\% Xe, 50\% methane atmosphere has an
intermediate extent (\(H = 3.17\) km).

    \subsection{Problem 2:}\label{problem-2}

\begin{quote}
{[}4 points total{]} A spherical hot-air balloon is designed to float at
the \textasciitilde{}1 bar, 60 K level in the atmosphere of Neptune.
Scientists hope to track the winds in the atmosphere of Neptune by
following the motion of the balloon. The balloon is also instrumented
with atmospheric sensors. The weight of the balloon and payload is 30
kg. What balloon volume and radius is required for a ``hot air''
temperature of 100 K? (Data: mean molar mass of Neptune atmosphere = 2.6
g/mol). {[}4 pts{]}
\end{quote}

Here's what we know:

    \begin{Verbatim}[commandchars=\\\{\}]
{\color{incolor}In [{\color{incolor}23}]:} \PY{n}{P2} \PY{o}{=} \PY{l+m+mf}{1.0} \PY{o}{*} \PY{n}{u}\PY{o}{.}\PY{n}{bar}\PY{p}{;} \PY{n}{P2}
\end{Verbatim}
\texttt{\color{outcolor}Out[{\color{outcolor}23}]:}
    
    \(1 \; \mathrm{bar}\)

    

    \begin{Verbatim}[commandchars=\\\{\}]
{\color{incolor}In [{\color{incolor}24}]:} \PY{n}{T2} \PY{o}{=} \PY{l+m+mf}{60.} \PY{o}{*} \PY{n}{u}\PY{o}{.}\PY{n}{K}\PY{p}{;} \PY{n}{T2}
\end{Verbatim}
\texttt{\color{outcolor}Out[{\color{outcolor}24}]:}
    
    \(60 \; \mathrm{K}\)

    

    \begin{Verbatim}[commandchars=\\\{\}]
{\color{incolor}In [{\color{incolor}25}]:} \PY{n}{m\PYZus{}payload} \PY{o}{=} \PY{l+m+mf}{30.0} \PY{o}{*} \PY{n}{u}\PY{o}{.}\PY{n}{kg}\PY{p}{;} \PY{n}{m\PYZus{}payload} 
\end{Verbatim}
\texttt{\color{outcolor}Out[{\color{outcolor}25}]:}
    
    \(30 \; \mathrm{kg}\)

    

    \begin{Verbatim}[commandchars=\\\{\}]
{\color{incolor}In [{\color{incolor}26}]:} \PY{n}{T\PYZus{}hot\PYZus{}air} \PY{o}{=} \PY{l+m+mf}{100.0} \PY{o}{*} \PY{n}{u}\PY{o}{.}\PY{n}{K}\PY{p}{;} \PY{n}{T\PYZus{}hot\PYZus{}air}
\end{Verbatim}
\texttt{\color{outcolor}Out[{\color{outcolor}26}]:}
    
    \(100 \; \mathrm{K}\)

    

    \begin{Verbatim}[commandchars=\\\{\}]
{\color{incolor}In [{\color{incolor}27}]:} \PY{n}{mu\PYZus{}Nep} \PY{o}{=} \PY{l+m+mf}{2.6} \PY{o}{*} \PY{n}{u}\PY{o}{.}\PY{n}{g}\PY{o}{/}\PY{n}{u}\PY{o}{.}\PY{n}{mol}\PY{p}{;} \PY{n}{mu\PYZus{}Nep}
\end{Verbatim}
\texttt{\color{outcolor}Out[{\color{outcolor}27}]:}
    
    \(2.6 \; \mathrm{\frac{g}{mol}}\)

    

    An object inbedded in another medium will float due to buoyancy if the
downward weight (\(mg\)) of the object is equal to the weight of the
displaced medium. This is Archimedes' principle. In other words, the net
force on the payload must be zero. From the notes we can write the
buoyancy force as,\\
\[ F_{buoy} = -g (\rho_{in} - \rho_{out}) V \]\\
which must balance the downward force of the balloon payload,\\
\[ F_{pay} = m_p g .\]\\
Setting \(F_{buoy} = F_{pay}\) and solving for the balloon volume, we
get\\
\[V = \frac{m_p}{\rho_{out} - \rho_{in}}\]

Ultimately it is the temperature difference between the gas inside the
balloon versus outside that drives the density difference that provides
the buoyant force. We'll calculate the density of an ideal gas:
\[ \rho = \frac{\bar{m} P }{k T} \]\\
Inside the balloon the density is:

    \begin{Verbatim}[commandchars=\\\{\}]
{\color{incolor}In [{\color{incolor}28}]:} \PY{n}{rho\PYZus{}in} \PY{o}{=} \PY{p}{(}\PY{p}{(}\PY{n}{mu\PYZus{}Nep} \PY{o}{/} \PY{n}{NA}\PY{p}{)} \PY{o}{*} \PY{n}{P2}\PY{o}{.}\PY{n}{decompose}\PY{p}{(}\PY{p}{)}\PY{p}{)} \PY{o}{/} \PY{p}{(}\PY{n}{k} \PY{o}{*} \PY{n}{T\PYZus{}hot\PYZus{}air}\PY{p}{)}\PY{p}{;} \PY{n}{rho\PYZus{}in} 
\end{Verbatim}
\texttt{\color{outcolor}Out[{\color{outcolor}28}]:}
    
    \(0.31270823 \; \mathrm{\frac{kg}{m^{3}}}\)

    

    and outside the balloon the density is:

    \begin{Verbatim}[commandchars=\\\{\}]
{\color{incolor}In [{\color{incolor}29}]:} \PY{n}{rho\PYZus{}out} \PY{o}{=} \PY{p}{(}\PY{p}{(}\PY{n}{mu\PYZus{}Nep} \PY{o}{/} \PY{n}{NA}\PY{p}{)} \PY{o}{*} \PY{n}{P2}\PY{o}{.}\PY{n}{decompose}\PY{p}{(}\PY{p}{)}\PY{p}{)} \PY{o}{/} \PY{p}{(}\PY{n}{k} \PY{o}{*} \PY{n}{T2}\PY{p}{)}\PY{p}{;} \PY{n}{rho\PYZus{}out}
\end{Verbatim}
\texttt{\color{outcolor}Out[{\color{outcolor}29}]:}
    
    \(0.52118038 \; \mathrm{\frac{kg}{m^{3}}}\)

    

    Now we'll solve for the balloon volume,

    \begin{Verbatim}[commandchars=\\\{\}]
{\color{incolor}In [{\color{incolor}30}]:} \PY{n}{V} \PY{o}{=} \PY{n}{m\PYZus{}payload} \PY{o}{/} \PY{p}{(}\PY{n}{rho\PYZus{}out} \PY{o}{\PYZhy{}} \PY{n}{rho\PYZus{}in}\PY{p}{)}\PY{p}{;} \PY{n}{V}
\end{Verbatim}
\texttt{\color{outcolor}Out[{\color{outcolor}30}]:}
    
    \(143.90411 \; \mathrm{m^{3}}\)

    

    And, assuming the balloon is spherical we can trivially find the radius:
\[ V = \frac{4}{3} \pi R^3 \rightarrow R = \left ( \frac{3V}{4 \pi} \right ) ^{1/3}\]

    \begin{Verbatim}[commandchars=\\\{\}]
{\color{incolor}In [{\color{incolor}31}]:} \PY{n}{R} \PY{o}{=} \PY{p}{(}\PY{p}{(}\PY{l+m+mf}{3.} \PY{o}{*} \PY{n}{V}\PY{p}{)}\PY{o}{/}\PY{p}{(}\PY{l+m+mf}{4.} \PY{o}{*} \PY{n}{np}\PY{o}{.}\PY{n}{pi}\PY{p}{)}\PY{p}{)}\PY{o}{*}\PY{o}{*}\PY{p}{(}\PY{l+m+mf}{1.}\PY{o}{/}\PY{l+m+mf}{3.}\PY{p}{)}\PY{p}{;} \PY{n}{R}
\end{Verbatim}
\texttt{\color{outcolor}Out[{\color{outcolor}31}]:}
    
    \(3.2508345 \; \mathrm{m}\)

    

    A 3.25-m radius balloon seems reasonably sized.

    \subsection{Problem 3:}\label{problem-3}

\begin{quote}
{[}6 points total{]} In deriving the hydrostatic equation for an
isothermal planetary atmosphere, we assumed that the scale height \(H\)
was much smaller than the radius of the planet; that is, that \(g\) can
be considered a constant throughout the atmosphere, as a reasonable
approximation. Suppose early Titan had much hydrogen in its atmosphere.
If so, our approximation is no longer valid because \(g = (GM)/r^2\) at
planetocentric distance \(r\).
\end{quote}

\begin{quote}
\begin{enumerate}
\def\labelenumi{(\roman{enumi})}
\tightlist
\item
  Re-derive the hydrostatic equation for this case, taking into account
  the variation of g with planetocentric distance. I suggest you start
  with \(dP / P = −(\bar{m}g / R\bar{T} )dr\) . {[}3 pts{]}\\
\item
  Assume that early Titan's atmosphere was made of pure hydrogen
  (H\(_2\)). What would be the scale height using the usual formula and
  assuming an isothermal temperature of 90 K? Look up Titan's radius. As
  a percentage, how much is the scale height a fraction of Titan's
  radius? {[}Titan \(g = 1.37\) m/s\(^2\){]} {[}2 pts{]}\\
\item
  Why is a pure hydrogen atmosphere unlikely to be found around a planet
  like Titan? {[}1 pt{]}
\end{enumerate}
\end{quote}

\subsubsection{Part 1:}\label{part-1}

Starting with the hydrostatic equation,\\
\[ \frac{dP}{P} = - \left ( \frac{\bar{m}g}{R\bar{T}} \right) dr \]\\
and plugging in \[ g = \frac{GM}{r^2} \] we get,\\
\[ \frac{dP}{P} = - \left ( \frac{GM\bar{m}}{R\bar{T}} \right ) \frac{dr}{r^2}\]\\
Now we want to integrate both sides of the above expression from the
surface (at \(z=R_p\)) to some height z,\\
\[ \int_{p_s}^{p(z)} \frac{dP}{P} = - \left ( \frac{GM\bar{m}}{R\bar{T}} \right ) \int_{R_p}^{z} \frac{dr}{r^2}\]\\
\[ \ln \left ( \frac{p(z)}{p_s} \right ) = \left ( \frac{GM\bar{m}}{R\bar{T}} \right ) \left ( \frac{1}{z} - \frac{1}{R_p} \right ) \]\\
Exponentiate both sides and simplifying our result,\\
\[ p(z) = p_s \exp \left ( \frac{-GM\bar{m}}{R_p R\bar{T}} \right ) \exp \left ( \frac{GM\bar{m}}{z R\bar{T}}  \right )
\]\\
If we define the surface gravity,\\
\[ g_s = \frac{GM}{R_p^2} \]\\
Then we can write the typical pressure scale height as\\
\[ H = \frac{R \bar{T}}{g_s} \]\\
Our solution now simplifies to:
\[ p(z) = p_s \exp \left ( \frac{-R_p}{H} \right ) \exp \left ( \frac{R_p^2}{H z}  \right ) \]

\subsubsection{Part 2:}\label{part-2}

We'll start with the isothermal temperature of

    \begin{Verbatim}[commandchars=\\\{\}]
{\color{incolor}In [{\color{incolor}32}]:} \PY{n}{T3} \PY{o}{=} \PY{l+m+mi}{90} \PY{o}{*} \PY{n}{u}\PY{o}{.}\PY{n}{K}\PY{p}{;} \PY{n}{T3}
\end{Verbatim}
\texttt{\color{outcolor}Out[{\color{outcolor}32}]:}
    
    \(90 \; \mathrm{K}\)

    

    The mean molecular weight for hydrogen gas:

    \begin{Verbatim}[commandchars=\\\{\}]
{\color{incolor}In [{\color{incolor}33}]:} \PY{n}{muH2} \PY{o}{=} \PY{l+m+mf}{2.0} \PY{o}{*} \PY{n}{u}\PY{o}{.}\PY{n}{g} \PY{o}{/} \PY{n}{u}\PY{o}{.}\PY{n}{mol}\PY{p}{;} \PY{n}{muH2}
\end{Verbatim}
\texttt{\color{outcolor}Out[{\color{outcolor}33}]:}
    
    \(2 \; \mathrm{\frac{g}{mol}}\)

    

    which, in grams, is

    \begin{Verbatim}[commandchars=\\\{\}]
{\color{incolor}In [{\color{incolor}34}]:} \PY{n}{mH2} \PY{o}{=} \PY{n}{muH2} \PY{o}{/} \PY{n}{NA}\PY{p}{;} \PY{n}{mH2}
\end{Verbatim}
\texttt{\color{outcolor}Out[{\color{outcolor}34}]:}
    
    \(3.3210781 \times 10^{-24} \; \mathrm{g}\)

    

    The gravity of Titan:

    \begin{Verbatim}[commandchars=\\\{\}]
{\color{incolor}In [{\color{incolor}35}]:} \PY{n}{g\PYZus{}titan} \PY{o}{=} \PY{l+m+mf}{1.37} \PY{o}{*} \PY{n}{u}\PY{o}{.}\PY{n}{m} \PY{o}{/} \PY{n}{u}\PY{o}{.}\PY{n}{second}\PY{o}{*}\PY{o}{*}\PY{l+m+mi}{2}\PY{p}{;} \PY{n}{g\PYZus{}titan}
\end{Verbatim}
\texttt{\color{outcolor}Out[{\color{outcolor}35}]:}
    
    \(1.37 \; \mathrm{\frac{m}{s^{2}}}\)

    

    We calculate the pressure scale height to be:

    \begin{Verbatim}[commandchars=\\\{\}]
{\color{incolor}In [{\color{incolor}36}]:} \PY{n}{H3} \PY{o}{=} \PY{n}{pressure\PYZus{}scale\PYZus{}height}\PY{p}{(}\PY{n}{T3}\PY{p}{,} \PY{n}{mH2}\PY{p}{,} \PY{n}{g\PYZus{}titan}\PY{p}{)} \PY{o}{*} \PY{n}{u}\PY{o}{.}\PY{n}{m}\PY{o}{.}\PY{n}{in\PYZus{}units}\PY{p}{(}\PY{n}{u}\PY{o}{.}\PY{n}{km}\PY{p}{)} \PY{o}{*} \PY{n}{u}\PY{o}{.}\PY{n}{km}\PY{o}{/}\PY{n}{u}\PY{o}{.}\PY{n}{m}\PY{p}{;} \PY{n}{H3}
\end{Verbatim}
\texttt{\color{outcolor}Out[{\color{outcolor}36}]:}
    
    \(273.1027 \; \mathrm{km}\)

    

    Titan's radius is

    \begin{Verbatim}[commandchars=\\\{\}]
{\color{incolor}In [{\color{incolor}37}]:} \PY{n}{R\PYZus{}titan} \PY{o}{=} \PY{l+m+mf}{2576.} \PY{o}{*} \PY{n}{u}\PY{o}{.}\PY{n}{km}\PY{p}{;} \PY{n}{R\PYZus{}titan}
\end{Verbatim}
\texttt{\color{outcolor}Out[{\color{outcolor}37}]:}
    
    \(2576 \; \mathrm{km}\)

    

    This means that if Titan ever had a pure hydrogen atmosphere, the
pressure scale height would have been

    \begin{Verbatim}[commandchars=\\\{\}]
{\color{incolor}In [{\color{incolor}38}]:} \PY{n}{H3} \PY{o}{/} \PY{n}{R\PYZus{}titan} \PY{o}{*} \PY{l+m+mi}{100} \PY{o}{*} \PY{n}{u}\PY{o}{.}\PY{n}{percent}
\end{Verbatim}
\texttt{\color{outcolor}Out[{\color{outcolor}38}]:}
    
    \(10.601813 \; \mathrm{\%}\)

    

    the radius of Titan.

    \subsubsection{Part 3:}\label{part-3}

A pure hydrogen atmosphere is unlikely to be found for a planet like
Titan because the low mass of hydrogen leads to such a vertically
extended atmosphere that the molecules can easily escape the low gravity
of the small world. Becuase temperature is the average kinetic energy of
the molecules in a gas, low mass molecules achieve much higher
velocities than heavier molecules (like carbon dioxide and hydrocarbons)
at any given temperature. Atmospheric escape can occur when the velocity
of molecules exceeds the escape velocity of a planet. This condition is
easily met for hydrogen in the atmospheres of small planets and moons.

    \subsection{Problem 4:}\label{problem-4}

\begin{quote}
{[}7 points{]} An Earth-based spectroscopic measurement of the strength
of some features in the infrared spectrum of Mars reports a column
abundance of CO2 of 8000 cm-amagat.
\end{quote}

\begin{quote}
\begin{enumerate}
\def\labelenumi{\alph{enumi}.}
\tightlist
\item
  What is the corresponding surface pressure of carbon dioxide in
  millibars? {[}Data: molar mass of CO2 is 44 g/mol; Martian gravity g =
  3.72 m s-2{]}. {[}4 pts{]}
\end{enumerate}
\end{quote}

\begin{quote}
\begin{enumerate}
\def\labelenumi{\alph{enumi}.}
\setcounter{enumi}{1}
\tightlist
\item
  The paper reporting this result does not bother to tell us whether the
  abundance refers to the sub-Earth region of Mars (i.e.~a vertical
  incidence region of Mars in the direct line-of-sight from Earth) or to
  an average over the entire disk of Mars. Would the surface pressure be
  less, the same, or greater than the value calculated in (a) if the
  reported abundance is actually a whole- disk average? Justify your
  answer in a qualitative fashion. (With a bit of thought and `one step'
  arithmetic, it is possible to calculate disk-average pressure versus
  line of sight pressure. You're looking to deduce a weighting factor
  for a disk average rather than a direct path perpendicular to the
  surface at the equator. This gets a bonus point, but if you're not
  sure, don't sweat it because it's subtle). {[}2 pts + 1 bonus pt{]}
\end{enumerate}
\end{quote}

    \subsubsection{Part 1:}\label{part-1}

Let's define the atm unit of pressure in terms of bars:

    \begin{Verbatim}[commandchars=\\\{\}]
{\color{incolor}In [{\color{incolor}43}]:} \PY{n}{atm} \PY{o}{=} \PY{l+m+mf}{1.01325} \PY{o}{*} \PY{n}{u}\PY{o}{.}\PY{n}{bar}\PY{p}{;} \PY{n}{atm}
\end{Verbatim}
\texttt{\color{outcolor}Out[{\color{outcolor}43}]:}
    
    \(1.01325 \; \mathrm{bar}\)

    

    Then the column abundance thing (\(\mathcal{N}\)) is

    \begin{Verbatim}[commandchars=\\\{\}]
{\color{incolor}In [{\color{incolor}44}]:} \PY{n}{N} \PY{o}{=} \PY{l+m+mf}{8000.} \PY{o}{*} \PY{n}{u}\PY{o}{.}\PY{n}{cm}\PY{o}{*} \PY{n}{atm} \PY{p}{;}\PY{n}{N}
\end{Verbatim}
\texttt{\color{outcolor}Out[{\color{outcolor}44}]:}
    
    \(8106 \; \mathrm{bar\,cm}\)

    

    The mean molecular weight of CO\(_2\) is

    \begin{Verbatim}[commandchars=\\\{\}]
{\color{incolor}In [{\color{incolor}45}]:} \PY{n}{muCO2} \PY{o}{=} \PY{l+m+mf}{44.} \PY{o}{*} \PY{n}{u}\PY{o}{.}\PY{n}{g} \PY{o}{/} \PY{n}{u}\PY{o}{.}\PY{n}{mol}\PY{p}{;} \PY{n}{muCO2}
\end{Verbatim}
\texttt{\color{outcolor}Out[{\color{outcolor}45}]:}
    
    \(44 \; \mathrm{\frac{g}{mol}}\)

    

    which is

    \begin{Verbatim}[commandchars=\\\{\}]
{\color{incolor}In [{\color{incolor}46}]:} \PY{n}{mCO2} \PY{o}{=} \PY{n}{muCO2} \PY{o}{/} \PY{n}{NA}\PY{p}{;} \PY{n}{mCO2}
\end{Verbatim}
\texttt{\color{outcolor}Out[{\color{outcolor}46}]:}
    
    \(7.3063718 \times 10^{-23} \; \mathrm{g}\)

    

    Finall, the gravity of Mars and the Loschmidt number are

    \begin{Verbatim}[commandchars=\\\{\}]
{\color{incolor}In [{\color{incolor}47}]:} \PY{n}{g\PYZus{}mars} \PY{o}{=} \PY{l+m+mf}{3.72} \PY{o}{*} \PY{n}{u}\PY{o}{.}\PY{n}{m} \PY{o}{/} \PY{n}{u}\PY{o}{.}\PY{n}{second}\PY{o}{*}\PY{o}{*}\PY{l+m+mi}{2}\PY{p}{;} \PY{n}{g\PYZus{}mars}
\end{Verbatim}
\texttt{\color{outcolor}Out[{\color{outcolor}47}]:}
    
    \(3.72 \; \mathrm{\frac{m}{s^{2}}}\)

    

    \begin{Verbatim}[commandchars=\\\{\}]
{\color{incolor}In [{\color{incolor}48}]:} \PY{n}{n0} \PY{o}{=} \PY{l+m+mf}{2.687e19} \PY{o}{*} \PY{n}{u}\PY{o}{.}\PY{n}{cm}\PY{o}{*}\PY{o}{*}\PY{o}{\PYZhy{}}\PY{l+m+mi}{3}\PY{p}{;} \PY{n}{n0}
\end{Verbatim}
\texttt{\color{outcolor}Out[{\color{outcolor}48}]:}
    
    \(2.687 \times 10^{19} \; \mathrm{\frac{1}{cm^{3}}}\)

    

    From the notes, we can calculate pressure from the column abundance,
atmospheric mass, and gravity using this relation\\
\[ p(z) = N_c(z) \bar{m} g \]\\
where true column abundance can be computed from via,\\
\[ N_c(z) = n_0 \mathcal{N} \]\\
This gives the following in fundamental SI unit bases:

    \begin{Verbatim}[commandchars=\\\{\}]
{\color{incolor}In [{\color{incolor}50}]:} \PY{n}{p\PYZus{}mars1} \PY{o}{=} \PY{n}{N}\PY{o}{.}\PY{n}{decompose}\PY{p}{(}\PY{p}{)} \PY{o}{*} \PY{n}{n0}\PY{o}{.}\PY{n}{decompose}\PY{p}{(}\PY{p}{)} \PY{o}{*} \PY{n}{mCO2}\PY{o}{.}\PY{n}{decompose}\PY{p}{(}\PY{p}{)} \PY{o}{*} \PY{n}{g\PYZus{}mars}\PY{o}{.}\PY{n}{decompose}\PY{p}{(}\PY{p}{)}\PY{p}{;} \PY{n}{p\PYZus{}mars1}
\end{Verbatim}
\texttt{\color{outcolor}Out[{\color{outcolor}50}]:}
    
    \(59199627 \; \mathrm{\frac{kg^{2}}{m^{2}\,s^{4}}}\)

    

    Note that the fundamental SI units of pressure are
\(\frac{\text{kg}}{\text{m s}^2}\), e.g.

    \begin{Verbatim}[commandchars=\\\{\}]
{\color{incolor}In [{\color{incolor}188}]:} \PY{n}{u}\PY{o}{.}\PY{n}{bar}\PY{o}{.}\PY{n}{decompose}\PY{p}{(}\PY{p}{)}
\end{Verbatim}
\texttt{\color{outcolor}Out[{\color{outcolor}188}]:}
    
    \(\mathrm{100000\,\frac{kg}{m\,s^{2}}}\)

    

    We have calculated something that has units of pressure squared!
Therefore, it would seems that this equation from the notes:

\[ p(z) = N_c(z) \bar{m} g \]

should actually be:

\[ p(z) = \sqrt{N_c(z) \bar{m} g} \]

Using this expression we get:

    \begin{Verbatim}[commandchars=\\\{\}]
{\color{incolor}In [{\color{incolor}53}]:} \PY{n}{p\PYZus{}mars2} \PY{o}{=} \PY{n}{np}\PY{o}{.}\PY{n}{sqrt}\PY{p}{(}\PY{p}{(}\PY{n}{n0} \PY{o}{*} \PY{n}{N}\PY{p}{)}\PY{o}{.}\PY{n}{decompose}\PY{p}{(}\PY{p}{)} \PY{o}{*} \PY{n}{g\PYZus{}mars}\PY{o}{.}\PY{n}{decompose}\PY{p}{(}\PY{p}{)} \PY{o}{*} \PY{n}{mCO2}\PY{o}{.}\PY{n}{decompose}\PY{p}{(}\PY{p}{)}\PY{p}{)}\PY{p}{;} \PY{n}{p\PYZus{}mars2}
\end{Verbatim}
\texttt{\color{outcolor}Out[{\color{outcolor}53}]:}
    
    \(7694.1294 \; \mathrm{\frac{kg}{m\,s^{2}}}\)

    

    Converting to bars, this is

    \begin{Verbatim}[commandchars=\\\{\}]
{\color{incolor}In [{\color{incolor}55}]:} \PY{n}{p\PYZus{}mars2\PYZus{}bar} \PY{o}{=} \PY{p}{(}\PY{n}{p\PYZus{}mars2} \PY{o}{/} \PY{n}{u}\PY{o}{.}\PY{n}{bar}\PY{o}{.}\PY{n}{decompose}\PY{p}{(}\PY{p}{)}\PY{p}{)}\PY{o}{.}\PY{n}{decompose}\PY{p}{(}\PY{p}{)} \PY{o}{*} \PY{n}{u}\PY{o}{.}\PY{n}{bar}\PY{p}{;} \PY{n}{p\PYZus{}mars2\PYZus{}bar}
\end{Verbatim}
\texttt{\color{outcolor}Out[{\color{outcolor}55}]:}
    
    \(0.076941294 \; \mathrm{bar}\)

    

    This is about 77 mbar of CO\(_2\) at the surface of Mars, roughly an
order of magnitude higher than the true surface pressure of Mars
(\({\sim}6\) mbar).

    \subsubsection{Part 2:}\label{part-2}

If the reported column abundance were disk-integrated, rather than the
vertical column at the sub-Earth point, the inferred surface pressure
would be lower. The sub-Earth path to the surface is the minimum
possible path over the ensemble of paths that connect the observer and
all visible locations of the surface of Mars. Integrating the column
abundance over the entire disk will necessarily include contributions
from these longer paths, which encounter more CO\(_2\) molecules, and
bias us towards higher surface pressures if we misinterpret it for only
the sub-observer view.

Since the true surface pressure of the CO\(_2\)-dominated Martian
atmosphere is \({\sim}6\) mbar, we are off by a factor of about 10. So
perhaps the weighting factor is 10 or, better yet, \(\pi^2\), since that
is the area of the disk in normalized units.

To rigorously solve for the difference between inferred surface
pressures using the nadir and disk-integrated views we must calculate
the disk-averaged path length to the surface. This requires some
geometry and a 2D integral over the visible disk of Mars. The answer is
probably \(\pi^2\).

    \begin{Verbatim}[commandchars=\\\{\}]
{\color{incolor}In [{\color{incolor} }]:} 
\end{Verbatim}


    % Add a bibliography block to the postdoc
    
    
    
    \end{document}
