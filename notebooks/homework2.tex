
% Default to the notebook output style

    


% Inherit from the specified cell style.




    
\documentclass[11pt]{article}

    
    
    \usepackage[T1]{fontenc}
    % Nicer default font than Computer Modern for most use cases
    \usepackage{palatino}

    % Basic figure setup, for now with no caption control since it's done
    % automatically by Pandoc (which extracts ![](path) syntax from Markdown).
    \usepackage{graphicx}
    % We will generate all images so they have a width \maxwidth. This means
    % that they will get their normal width if they fit onto the page, but
    % are scaled down if they would overflow the margins.
    \makeatletter
    \def\maxwidth{\ifdim\Gin@nat@width>\linewidth\linewidth
    \else\Gin@nat@width\fi}
    \makeatother
    \let\Oldincludegraphics\includegraphics
    % Set max figure width to be 80% of text width, for now hardcoded.
    \renewcommand{\includegraphics}[1]{\Oldincludegraphics[width=.8\maxwidth]{#1}}
    % Ensure that by default, figures have no caption (until we provide a
    % proper Figure object with a Caption API and a way to capture that
    % in the conversion process - todo).
    \usepackage{caption}
    \DeclareCaptionLabelFormat{nolabel}{}
    \captionsetup{labelformat=nolabel}

    \usepackage{adjustbox} % Used to constrain images to a maximum size 
    \usepackage{xcolor} % Allow colors to be defined
    \usepackage{enumerate} % Needed for markdown enumerations to work
    \usepackage{geometry} % Used to adjust the document margins
    \usepackage{amsmath} % Equations
    \usepackage{amssymb} % Equations
    \usepackage{textcomp} % defines textquotesingle
    % Hack from http://tex.stackexchange.com/a/47451/13684:
    \AtBeginDocument{%
        \def\PYZsq{\textquotesingle}% Upright quotes in Pygmentized code
    }
    \usepackage{upquote} % Upright quotes for verbatim code
    \usepackage{eurosym} % defines \euro
    \usepackage[mathletters]{ucs} % Extended unicode (utf-8) support
    \usepackage[utf8x]{inputenc} % Allow utf-8 characters in the tex document
    \usepackage{fancyvrb} % verbatim replacement that allows latex
    \usepackage{grffile} % extends the file name processing of package graphics 
                         % to support a larger range 
    % The hyperref package gives us a pdf with properly built
    % internal navigation ('pdf bookmarks' for the table of contents,
    % internal cross-reference links, web links for URLs, etc.)
    \usepackage{hyperref}
    \usepackage{longtable} % longtable support required by pandoc >1.10
    \usepackage{booktabs}  % table support for pandoc > 1.12.2
    \usepackage[normalem]{ulem} % ulem is needed to support strikethroughs (\sout)
                                % normalem makes italics be italics, not underlines
    

    
    
    % Colors for the hyperref package
    \definecolor{urlcolor}{rgb}{0,.145,.698}
    \definecolor{linkcolor}{rgb}{.71,0.21,0.01}
    \definecolor{citecolor}{rgb}{.12,.54,.11}

    % ANSI colors
    \definecolor{ansi-black}{HTML}{3E424D}
    \definecolor{ansi-black-intense}{HTML}{282C36}
    \definecolor{ansi-red}{HTML}{E75C58}
    \definecolor{ansi-red-intense}{HTML}{B22B31}
    \definecolor{ansi-green}{HTML}{00A250}
    \definecolor{ansi-green-intense}{HTML}{007427}
    \definecolor{ansi-yellow}{HTML}{DDB62B}
    \definecolor{ansi-yellow-intense}{HTML}{B27D12}
    \definecolor{ansi-blue}{HTML}{208FFB}
    \definecolor{ansi-blue-intense}{HTML}{0065CA}
    \definecolor{ansi-magenta}{HTML}{D160C4}
    \definecolor{ansi-magenta-intense}{HTML}{A03196}
    \definecolor{ansi-cyan}{HTML}{60C6C8}
    \definecolor{ansi-cyan-intense}{HTML}{258F8F}
    \definecolor{ansi-white}{HTML}{C5C1B4}
    \definecolor{ansi-white-intense}{HTML}{A1A6B2}

    % commands and environments needed by pandoc snippets
    % extracted from the output of `pandoc -s`
    \providecommand{\tightlist}{%
      \setlength{\itemsep}{0pt}\setlength{\parskip}{0pt}}
    \DefineVerbatimEnvironment{Highlighting}{Verbatim}{commandchars=\\\{\}}
    % Add ',fontsize=\small' for more characters per line
    \newenvironment{Shaded}{}{}
    \newcommand{\KeywordTok}[1]{\textcolor[rgb]{0.00,0.44,0.13}{\textbf{{#1}}}}
    \newcommand{\DataTypeTok}[1]{\textcolor[rgb]{0.56,0.13,0.00}{{#1}}}
    \newcommand{\DecValTok}[1]{\textcolor[rgb]{0.25,0.63,0.44}{{#1}}}
    \newcommand{\BaseNTok}[1]{\textcolor[rgb]{0.25,0.63,0.44}{{#1}}}
    \newcommand{\FloatTok}[1]{\textcolor[rgb]{0.25,0.63,0.44}{{#1}}}
    \newcommand{\CharTok}[1]{\textcolor[rgb]{0.25,0.44,0.63}{{#1}}}
    \newcommand{\StringTok}[1]{\textcolor[rgb]{0.25,0.44,0.63}{{#1}}}
    \newcommand{\CommentTok}[1]{\textcolor[rgb]{0.38,0.63,0.69}{\textit{{#1}}}}
    \newcommand{\OtherTok}[1]{\textcolor[rgb]{0.00,0.44,0.13}{{#1}}}
    \newcommand{\AlertTok}[1]{\textcolor[rgb]{1.00,0.00,0.00}{\textbf{{#1}}}}
    \newcommand{\FunctionTok}[1]{\textcolor[rgb]{0.02,0.16,0.49}{{#1}}}
    \newcommand{\RegionMarkerTok}[1]{{#1}}
    \newcommand{\ErrorTok}[1]{\textcolor[rgb]{1.00,0.00,0.00}{\textbf{{#1}}}}
    \newcommand{\NormalTok}[1]{{#1}}
    
    % Additional commands for more recent versions of Pandoc
    \newcommand{\ConstantTok}[1]{\textcolor[rgb]{0.53,0.00,0.00}{{#1}}}
    \newcommand{\SpecialCharTok}[1]{\textcolor[rgb]{0.25,0.44,0.63}{{#1}}}
    \newcommand{\VerbatimStringTok}[1]{\textcolor[rgb]{0.25,0.44,0.63}{{#1}}}
    \newcommand{\SpecialStringTok}[1]{\textcolor[rgb]{0.73,0.40,0.53}{{#1}}}
    \newcommand{\ImportTok}[1]{{#1}}
    \newcommand{\DocumentationTok}[1]{\textcolor[rgb]{0.73,0.13,0.13}{\textit{{#1}}}}
    \newcommand{\AnnotationTok}[1]{\textcolor[rgb]{0.38,0.63,0.69}{\textbf{\textit{{#1}}}}}
    \newcommand{\CommentVarTok}[1]{\textcolor[rgb]{0.38,0.63,0.69}{\textbf{\textit{{#1}}}}}
    \newcommand{\VariableTok}[1]{\textcolor[rgb]{0.10,0.09,0.49}{{#1}}}
    \newcommand{\ControlFlowTok}[1]{\textcolor[rgb]{0.00,0.44,0.13}{\textbf{{#1}}}}
    \newcommand{\OperatorTok}[1]{\textcolor[rgb]{0.40,0.40,0.40}{{#1}}}
    \newcommand{\BuiltInTok}[1]{{#1}}
    \newcommand{\ExtensionTok}[1]{{#1}}
    \newcommand{\PreprocessorTok}[1]{\textcolor[rgb]{0.74,0.48,0.00}{{#1}}}
    \newcommand{\AttributeTok}[1]{\textcolor[rgb]{0.49,0.56,0.16}{{#1}}}
    \newcommand{\InformationTok}[1]{\textcolor[rgb]{0.38,0.63,0.69}{\textbf{\textit{{#1}}}}}
    \newcommand{\WarningTok}[1]{\textcolor[rgb]{0.38,0.63,0.69}{\textbf{\textit{{#1}}}}}
    
    
    % Define a nice break command that doesn't care if a line doesn't already
    % exist.
    \def\br{\hspace*{\fill} \\* }
    % Math Jax compatability definitions
    \def\gt{>}
    \def\lt{<}
    % Document parameters
    \title{homework2}
    
    
    

    % Pygments definitions
    
\makeatletter
\def\PY@reset{\let\PY@it=\relax \let\PY@bf=\relax%
    \let\PY@ul=\relax \let\PY@tc=\relax%
    \let\PY@bc=\relax \let\PY@ff=\relax}
\def\PY@tok#1{\csname PY@tok@#1\endcsname}
\def\PY@toks#1+{\ifx\relax#1\empty\else%
    \PY@tok{#1}\expandafter\PY@toks\fi}
\def\PY@do#1{\PY@bc{\PY@tc{\PY@ul{%
    \PY@it{\PY@bf{\PY@ff{#1}}}}}}}
\def\PY#1#2{\PY@reset\PY@toks#1+\relax+\PY@do{#2}}

\expandafter\def\csname PY@tok@gd\endcsname{\def\PY@tc##1{\textcolor[rgb]{0.63,0.00,0.00}{##1}}}
\expandafter\def\csname PY@tok@gu\endcsname{\let\PY@bf=\textbf\def\PY@tc##1{\textcolor[rgb]{0.50,0.00,0.50}{##1}}}
\expandafter\def\csname PY@tok@gt\endcsname{\def\PY@tc##1{\textcolor[rgb]{0.00,0.27,0.87}{##1}}}
\expandafter\def\csname PY@tok@gs\endcsname{\let\PY@bf=\textbf}
\expandafter\def\csname PY@tok@gr\endcsname{\def\PY@tc##1{\textcolor[rgb]{1.00,0.00,0.00}{##1}}}
\expandafter\def\csname PY@tok@cm\endcsname{\let\PY@it=\textit\def\PY@tc##1{\textcolor[rgb]{0.25,0.50,0.50}{##1}}}
\expandafter\def\csname PY@tok@vg\endcsname{\def\PY@tc##1{\textcolor[rgb]{0.10,0.09,0.49}{##1}}}
\expandafter\def\csname PY@tok@vi\endcsname{\def\PY@tc##1{\textcolor[rgb]{0.10,0.09,0.49}{##1}}}
\expandafter\def\csname PY@tok@mh\endcsname{\def\PY@tc##1{\textcolor[rgb]{0.40,0.40,0.40}{##1}}}
\expandafter\def\csname PY@tok@cs\endcsname{\let\PY@it=\textit\def\PY@tc##1{\textcolor[rgb]{0.25,0.50,0.50}{##1}}}
\expandafter\def\csname PY@tok@ge\endcsname{\let\PY@it=\textit}
\expandafter\def\csname PY@tok@vc\endcsname{\def\PY@tc##1{\textcolor[rgb]{0.10,0.09,0.49}{##1}}}
\expandafter\def\csname PY@tok@il\endcsname{\def\PY@tc##1{\textcolor[rgb]{0.40,0.40,0.40}{##1}}}
\expandafter\def\csname PY@tok@go\endcsname{\def\PY@tc##1{\textcolor[rgb]{0.53,0.53,0.53}{##1}}}
\expandafter\def\csname PY@tok@cp\endcsname{\def\PY@tc##1{\textcolor[rgb]{0.74,0.48,0.00}{##1}}}
\expandafter\def\csname PY@tok@gi\endcsname{\def\PY@tc##1{\textcolor[rgb]{0.00,0.63,0.00}{##1}}}
\expandafter\def\csname PY@tok@gh\endcsname{\let\PY@bf=\textbf\def\PY@tc##1{\textcolor[rgb]{0.00,0.00,0.50}{##1}}}
\expandafter\def\csname PY@tok@ni\endcsname{\let\PY@bf=\textbf\def\PY@tc##1{\textcolor[rgb]{0.60,0.60,0.60}{##1}}}
\expandafter\def\csname PY@tok@nl\endcsname{\def\PY@tc##1{\textcolor[rgb]{0.63,0.63,0.00}{##1}}}
\expandafter\def\csname PY@tok@nn\endcsname{\let\PY@bf=\textbf\def\PY@tc##1{\textcolor[rgb]{0.00,0.00,1.00}{##1}}}
\expandafter\def\csname PY@tok@no\endcsname{\def\PY@tc##1{\textcolor[rgb]{0.53,0.00,0.00}{##1}}}
\expandafter\def\csname PY@tok@na\endcsname{\def\PY@tc##1{\textcolor[rgb]{0.49,0.56,0.16}{##1}}}
\expandafter\def\csname PY@tok@nb\endcsname{\def\PY@tc##1{\textcolor[rgb]{0.00,0.50,0.00}{##1}}}
\expandafter\def\csname PY@tok@nc\endcsname{\let\PY@bf=\textbf\def\PY@tc##1{\textcolor[rgb]{0.00,0.00,1.00}{##1}}}
\expandafter\def\csname PY@tok@nd\endcsname{\def\PY@tc##1{\textcolor[rgb]{0.67,0.13,1.00}{##1}}}
\expandafter\def\csname PY@tok@ne\endcsname{\let\PY@bf=\textbf\def\PY@tc##1{\textcolor[rgb]{0.82,0.25,0.23}{##1}}}
\expandafter\def\csname PY@tok@nf\endcsname{\def\PY@tc##1{\textcolor[rgb]{0.00,0.00,1.00}{##1}}}
\expandafter\def\csname PY@tok@si\endcsname{\let\PY@bf=\textbf\def\PY@tc##1{\textcolor[rgb]{0.73,0.40,0.53}{##1}}}
\expandafter\def\csname PY@tok@s2\endcsname{\def\PY@tc##1{\textcolor[rgb]{0.73,0.13,0.13}{##1}}}
\expandafter\def\csname PY@tok@nt\endcsname{\let\PY@bf=\textbf\def\PY@tc##1{\textcolor[rgb]{0.00,0.50,0.00}{##1}}}
\expandafter\def\csname PY@tok@nv\endcsname{\def\PY@tc##1{\textcolor[rgb]{0.10,0.09,0.49}{##1}}}
\expandafter\def\csname PY@tok@s1\endcsname{\def\PY@tc##1{\textcolor[rgb]{0.73,0.13,0.13}{##1}}}
\expandafter\def\csname PY@tok@ch\endcsname{\let\PY@it=\textit\def\PY@tc##1{\textcolor[rgb]{0.25,0.50,0.50}{##1}}}
\expandafter\def\csname PY@tok@m\endcsname{\def\PY@tc##1{\textcolor[rgb]{0.40,0.40,0.40}{##1}}}
\expandafter\def\csname PY@tok@gp\endcsname{\let\PY@bf=\textbf\def\PY@tc##1{\textcolor[rgb]{0.00,0.00,0.50}{##1}}}
\expandafter\def\csname PY@tok@sh\endcsname{\def\PY@tc##1{\textcolor[rgb]{0.73,0.13,0.13}{##1}}}
\expandafter\def\csname PY@tok@ow\endcsname{\let\PY@bf=\textbf\def\PY@tc##1{\textcolor[rgb]{0.67,0.13,1.00}{##1}}}
\expandafter\def\csname PY@tok@sx\endcsname{\def\PY@tc##1{\textcolor[rgb]{0.00,0.50,0.00}{##1}}}
\expandafter\def\csname PY@tok@bp\endcsname{\def\PY@tc##1{\textcolor[rgb]{0.00,0.50,0.00}{##1}}}
\expandafter\def\csname PY@tok@c1\endcsname{\let\PY@it=\textit\def\PY@tc##1{\textcolor[rgb]{0.25,0.50,0.50}{##1}}}
\expandafter\def\csname PY@tok@o\endcsname{\def\PY@tc##1{\textcolor[rgb]{0.40,0.40,0.40}{##1}}}
\expandafter\def\csname PY@tok@kc\endcsname{\let\PY@bf=\textbf\def\PY@tc##1{\textcolor[rgb]{0.00,0.50,0.00}{##1}}}
\expandafter\def\csname PY@tok@c\endcsname{\let\PY@it=\textit\def\PY@tc##1{\textcolor[rgb]{0.25,0.50,0.50}{##1}}}
\expandafter\def\csname PY@tok@mf\endcsname{\def\PY@tc##1{\textcolor[rgb]{0.40,0.40,0.40}{##1}}}
\expandafter\def\csname PY@tok@err\endcsname{\def\PY@bc##1{\setlength{\fboxsep}{0pt}\fcolorbox[rgb]{1.00,0.00,0.00}{1,1,1}{\strut ##1}}}
\expandafter\def\csname PY@tok@mb\endcsname{\def\PY@tc##1{\textcolor[rgb]{0.40,0.40,0.40}{##1}}}
\expandafter\def\csname PY@tok@ss\endcsname{\def\PY@tc##1{\textcolor[rgb]{0.10,0.09,0.49}{##1}}}
\expandafter\def\csname PY@tok@sr\endcsname{\def\PY@tc##1{\textcolor[rgb]{0.73,0.40,0.53}{##1}}}
\expandafter\def\csname PY@tok@mo\endcsname{\def\PY@tc##1{\textcolor[rgb]{0.40,0.40,0.40}{##1}}}
\expandafter\def\csname PY@tok@kd\endcsname{\let\PY@bf=\textbf\def\PY@tc##1{\textcolor[rgb]{0.00,0.50,0.00}{##1}}}
\expandafter\def\csname PY@tok@mi\endcsname{\def\PY@tc##1{\textcolor[rgb]{0.40,0.40,0.40}{##1}}}
\expandafter\def\csname PY@tok@kn\endcsname{\let\PY@bf=\textbf\def\PY@tc##1{\textcolor[rgb]{0.00,0.50,0.00}{##1}}}
\expandafter\def\csname PY@tok@cpf\endcsname{\let\PY@it=\textit\def\PY@tc##1{\textcolor[rgb]{0.25,0.50,0.50}{##1}}}
\expandafter\def\csname PY@tok@kr\endcsname{\let\PY@bf=\textbf\def\PY@tc##1{\textcolor[rgb]{0.00,0.50,0.00}{##1}}}
\expandafter\def\csname PY@tok@s\endcsname{\def\PY@tc##1{\textcolor[rgb]{0.73,0.13,0.13}{##1}}}
\expandafter\def\csname PY@tok@kp\endcsname{\def\PY@tc##1{\textcolor[rgb]{0.00,0.50,0.00}{##1}}}
\expandafter\def\csname PY@tok@w\endcsname{\def\PY@tc##1{\textcolor[rgb]{0.73,0.73,0.73}{##1}}}
\expandafter\def\csname PY@tok@kt\endcsname{\def\PY@tc##1{\textcolor[rgb]{0.69,0.00,0.25}{##1}}}
\expandafter\def\csname PY@tok@sc\endcsname{\def\PY@tc##1{\textcolor[rgb]{0.73,0.13,0.13}{##1}}}
\expandafter\def\csname PY@tok@sb\endcsname{\def\PY@tc##1{\textcolor[rgb]{0.73,0.13,0.13}{##1}}}
\expandafter\def\csname PY@tok@k\endcsname{\let\PY@bf=\textbf\def\PY@tc##1{\textcolor[rgb]{0.00,0.50,0.00}{##1}}}
\expandafter\def\csname PY@tok@se\endcsname{\let\PY@bf=\textbf\def\PY@tc##1{\textcolor[rgb]{0.73,0.40,0.13}{##1}}}
\expandafter\def\csname PY@tok@sd\endcsname{\let\PY@it=\textit\def\PY@tc##1{\textcolor[rgb]{0.73,0.13,0.13}{##1}}}

\def\PYZbs{\char`\\}
\def\PYZus{\char`\_}
\def\PYZob{\char`\{}
\def\PYZcb{\char`\}}
\def\PYZca{\char`\^}
\def\PYZam{\char`\&}
\def\PYZlt{\char`\<}
\def\PYZgt{\char`\>}
\def\PYZsh{\char`\#}
\def\PYZpc{\char`\%}
\def\PYZdl{\char`\$}
\def\PYZhy{\char`\-}
\def\PYZsq{\char`\'}
\def\PYZdq{\char`\"}
\def\PYZti{\char`\~}
% for compatibility with earlier versions
\def\PYZat{@}
\def\PYZlb{[}
\def\PYZrb{]}
\makeatother


    % Exact colors from NB
    \definecolor{incolor}{rgb}{0.0, 0.0, 0.5}
    \definecolor{outcolor}{rgb}{0.545, 0.0, 0.0}



    
    % Prevent overflowing lines due to hard-to-break entities
    \sloppy 
    % Setup hyperref package
    \hypersetup{
      breaklinks=true,  % so long urls are correctly broken across lines
      colorlinks=true,
      urlcolor=urlcolor,
      linkcolor=linkcolor,
      citecolor=citecolor,
      }
    % Slightly bigger margins than the latex defaults
    
    \geometry{verbose,tmargin=1in,bmargin=1in,lmargin=1in,rmargin=1in}
    
    

    \begin{document}
    
    
    \maketitle
    
    

    
    \begin{Verbatim}[commandchars=\\\{\}]
{\color{incolor}In [{\color{incolor}86}]:} \PY{c+c1}{\PYZsh{} Importing useful python packages}
         \PY{k+kn}{import} \PY{n+nn}{astropy.units} \PY{k+kn}{as} \PY{n+nn}{u}
         \PY{k+kn}{import} \PY{n+nn}{numpy} \PY{k+kn}{as} \PY{n+nn}{np}
\end{Verbatim}

    \subsection{Problem 1:}\label{problem-1}

\begin{quote}
You are asked to be part of a NASA team designing an ultraviolet
spectrometer to operate on the surface of Mars in the future. The
spectrometer will look vertically upward and measure the flux over the
wavelength range 180 nm to 400 nm in spectral intervals
\textasciitilde{}1 nm wide.
\end{quote}

\begin{quote}
\begin{enumerate}
\def\labelenumi{(\alph{enumi})}
\tightlist
\item
  During an early team meeting, a question is raised about whether the
  spectrometer could be made sensitive enough to measure the column
  abundance of ozone, even in the low latitudes where the O3 abundance
  is generally low. You know that the column abundance of O3 has been
  measured remotely and is estimated as
  \textasciitilde{}\(5 \times 10^{15}\) molecules cm-2 in the Martian
  low-latitudes. Meanwhile, a handbook gives the O3 absorption
  cross-section as 10-21 m2 at the deepest point in the Hartley bands,
  which occurs around a wavelength of 255 nm. You also know that as a
  rule-of- thumb, in order to discern a signal you need at least ten
  times better resolution than the signal itself. Estimate a minimum
  signal-to-noise ratio required by the UV spectrometer to measure
  column O3 abundance in the low latitudes of Mars with an overhead Sun.
  {[}3 pts{]}{[}Hint: Part (a) is not concerned with spectral resolution
  because O3 absorption is very broad. It's asking `how precisely do you
  need to measure the flux at 255 nm to determine the absorption by
  overhead ozone?'{]}
\end{enumerate}
\end{quote}

\begin{quote}
\begin{enumerate}
\def\labelenumi{(\alph{enumi})}
\setcounter{enumi}{1}
\tightlist
\item
  While thinking about the problem in (a), you realize from the expected
  UV spectrum on Mars could also be used in another way: the UV
  spectrometer could effectively act as a barometer, measuring the
  column abundance of CO2. Discuss what characteristic of the UV
  spectrum on Mars could be used to do this? {[}3 pts{]}{[}This question
  is an account of a real-life situation encountered by your
  instructor.{]}
\end{enumerate}
\end{quote}

\subsubsection{Answer to a)}\label{answer-to-a}

    \begin{Verbatim}[commandchars=\\\{\}]
{\color{incolor}In [{\color{incolor}3}]:} \PY{n}{lam} \PY{o}{=} \PY{l+m+mi}{255} \PY{o}{*} \PY{n}{u}\PY{o}{.}\PY{n}{nm}\PY{p}{;} \PY{n}{lam}
\end{Verbatim}
\texttt{\color{outcolor}Out[{\color{outcolor}3}]:}
    
    \(255 \; \mathrm{nm}\)

    

    \begin{Verbatim}[commandchars=\\\{\}]
{\color{incolor}In [{\color{incolor}4}]:} \PY{n}{Nc} \PY{o}{=} \PY{l+m+mf}{5e15} \PY{o}{*} \PY{n}{u}\PY{o}{.}\PY{n}{cm}\PY{o}{*}\PY{o}{*}\PY{o}{\PYZhy{}}\PY{l+m+mi}{2}\PY{p}{;} \PY{n}{Nc}
\end{Verbatim}
\texttt{\color{outcolor}Out[{\color{outcolor}4}]:}
    
    \(5 \times 10^{15} \; \mathrm{\frac{1}{cm^{2}}}\)

    

    \begin{Verbatim}[commandchars=\\\{\}]
{\color{incolor}In [{\color{incolor}5}]:} \PY{n}{sigma} \PY{o}{=} \PY{l+m+mf}{1e\PYZhy{}21} \PY{o}{*} \PY{n}{u}\PY{o}{.}\PY{n}{m}\PY{o}{*}\PY{o}{*}\PY{l+m+mi}{2}\PY{p}{;} \PY{n}{sigma}
\end{Verbatim}
\texttt{\color{outcolor}Out[{\color{outcolor}5}]:}
    
    \(1 \times 10^{-21} \; \mathrm{m^{2}}\)

    

    \textbf{The optical depth is just \(\tau = N \sigma\).}

    \begin{Verbatim}[commandchars=\\\{\}]
{\color{incolor}In [{\color{incolor}10}]:} \PY{n}{tau} \PY{o}{=} \PY{p}{(}\PY{n}{sigma} \PY{o}{*} \PY{n}{Nc}\PY{p}{)}\PY{o}{.}\PY{n}{decompose}\PY{p}{(}\PY{p}{)}\PY{p}{;} \PY{n}{tau}
\end{Verbatim}
\texttt{\color{outcolor}Out[{\color{outcolor}10}]:}
    
    \(0.05 \; \mathrm{}\)

    

    \textbf{Then the fractional reduction in flux is
\(\frac{I}{I_0} = e^{-\tau}\)}

    \begin{Verbatim}[commandchars=\\\{\}]
{\color{incolor}In [{\color{incolor}81}]:} \PY{n}{np}\PY{o}{.}\PY{n}{exp}\PY{p}{(}\PY{o}{\PYZhy{}}\PY{n}{tau}\PY{p}{)}
\end{Verbatim}

            \begin{Verbatim}[commandchars=\\\{\}]
{\color{outcolor}Out[{\color{outcolor}81}]:} 0.60638206237827741
\end{Verbatim}
        
    \textbf{So we would need a signal-to-noise ratio of 6.06 to measure to
ozone column abundance.}

    \subsubsection{Answer to b)}\label{answer-to-b}

\textbf{CO2 absorbs in the UV at wavelengths below about 220 nm. If we
were to build a UV spectrometer to access these wavelengths, similar to
ozone above, we could measure the column abundance of CO2. However,
since CO2 is the bulk atmospheric constituent, the partial pressure of
CO2 on Mars should be extremely close to the atmospheric pressure.
Therefore, UV measurements of the CO2 column abundance could be used as
a barometer. The assumption that the CO2 partial pressure is equal to
the atmospheric pressure breaks down if there are other gases in the
atmomphere with an apprciable abundance. For instance, seasonal
variations of water in the Martian atmosphere could corrupt such
inferences.}

    \subsection{Problem 2:}\label{problem-2}

\begin{quote}
The eccentricity of Mars' orbit is 0.0934, while the eccentricity of
Earth's orbit is 0.0167. How much greater, as a percentage, is the
equilibrium temperature of Mars and Earth at perihelion compared to
aphelion for each planet? {[}4 points{]}
\end{quote}

\subsubsection{Answer}\label{answer}

\textbf{Relative to the semi-major axis \(a\) and eccentricity \(e\),
the distance from the planet to the star at perihelion is
\[ r_p = a(1-e) \]\\
and at aphelion is \[ r_a = a(1+e) \]\\
For Earth and Mars these values are:}

    \begin{Verbatim}[commandchars=\\\{\}]
{\color{incolor}In [{\color{incolor}11}]:} \PY{n}{a\PYZus{}earth} \PY{o}{=} \PY{l+m+mf}{1.0}
         \PY{n}{a\PYZus{}mars} \PY{o}{=} \PY{l+m+mf}{1.5236}
         
         \PY{n}{e\PYZus{}earth} \PY{o}{=} \PY{l+m+mf}{0.0167}
         \PY{n}{e\PYZus{}mars} \PY{o}{=} \PY{l+m+mf}{0.0934}
         
         \PY{n}{rp\PYZus{}earth} \PY{o}{=} \PY{n}{a\PYZus{}earth} \PY{o}{*} \PY{p}{(}\PY{l+m+mf}{1.} \PY{o}{\PYZhy{}} \PY{n}{e\PYZus{}earth}\PY{p}{)}
         \PY{n}{ra\PYZus{}earth} \PY{o}{=} \PY{n}{a\PYZus{}earth} \PY{o}{*} \PY{p}{(}\PY{l+m+mf}{1.} \PY{o}{+} \PY{n}{e\PYZus{}earth}\PY{p}{)}
         
         \PY{n}{rp\PYZus{}mars} \PY{o}{=} \PY{n}{a\PYZus{}mars} \PY{o}{*} \PY{p}{(}\PY{l+m+mf}{1.} \PY{o}{\PYZhy{}} \PY{n}{e\PYZus{}mars}\PY{p}{)}
         \PY{n}{ra\PYZus{}mars} \PY{o}{=} \PY{n}{a\PYZus{}mars} \PY{o}{*} \PY{p}{(}\PY{l+m+mf}{1.} \PY{o}{+} \PY{n}{e\PYZus{}mars}\PY{p}{)}
\end{Verbatim}

    \begin{Verbatim}[commandchars=\\\{\}]
{\color{incolor}In [{\color{incolor}36}]:} \PY{k}{print} \PY{l+s+s2}{\PYZdq{}}\PY{l+s+s2}{Semi\PYZhy{}major axis   }\PY{l+s+s2}{\PYZdq{}}\PY{p}{,} \PY{l+s+s2}{\PYZdq{}}\PY{l+s+s2}{Perihelion   }\PY{l+s+s2}{\PYZdq{}}\PY{p}{,} \PY{l+s+s2}{\PYZdq{}}\PY{l+s+s2}{Aphelion}\PY{l+s+s2}{\PYZdq{}}
         \PY{k}{print} \PY{l+s+s2}{\PYZdq{}}\PY{l+s+s2}{\PYZhy{}\PYZhy{}\PYZhy{}\PYZhy{}\PYZhy{}\PYZhy{}\PYZhy{}\PYZhy{}\PYZhy{}\PYZhy{}\PYZhy{}\PYZhy{}\PYZhy{}\PYZhy{}\PYZhy{}   }\PY{l+s+s2}{\PYZdq{}}\PY{p}{,} \PY{l+s+s2}{\PYZdq{}}\PY{l+s+s2}{\PYZhy{}\PYZhy{}\PYZhy{}\PYZhy{}\PYZhy{}\PYZhy{}\PYZhy{}\PYZhy{}\PYZhy{}\PYZhy{}   }\PY{l+s+s2}{\PYZdq{}}\PY{p}{,} \PY{l+s+s2}{\PYZdq{}}\PY{l+s+s2}{\PYZhy{}\PYZhy{}\PYZhy{}\PYZhy{}\PYZhy{}\PYZhy{}\PYZhy{}\PYZhy{}}\PY{l+s+s2}{\PYZdq{}}
         \PY{k}{print} \PY{l+s+s2}{\PYZdq{}}\PY{l+s+si}{\PYZpc{}.4f}\PY{l+s+s2}{             }\PY{l+s+si}{\PYZpc{}.4f}\PY{l+s+s2}{        }\PY{l+s+si}{\PYZpc{}.4f}\PY{l+s+s2}{\PYZdq{}} \PY{o}{\PYZpc{}}\PY{p}{(}\PY{n}{a\PYZus{}earth}\PY{p}{,} \PY{n}{rp\PYZus{}earth}\PY{p}{,} \PY{n}{ra\PYZus{}earth}\PY{p}{)}
         \PY{k}{print} \PY{l+s+s2}{\PYZdq{}}\PY{l+s+si}{\PYZpc{}.4f}\PY{l+s+s2}{             }\PY{l+s+si}{\PYZpc{}.4f}\PY{l+s+s2}{        }\PY{l+s+si}{\PYZpc{}.4f}\PY{l+s+s2}{\PYZdq{}} \PY{o}{\PYZpc{}}\PY{p}{(}\PY{n}{a\PYZus{}mars}\PY{p}{,} \PY{n}{rp\PYZus{}mars}\PY{p}{,} \PY{n}{ra\PYZus{}mars}\PY{p}{)}
\end{Verbatim}

    \begin{Verbatim}[commandchars=\\\{\}]
Semi-major axis    Perihelion    Aphelion
---------------    ----------    --------
1.0000             0.9833        1.0167
1.5236             1.3813        1.6659

    \end{Verbatim}

    \textbf{Since the equilibrium temperature scales with heliocentric
distance as\\
\[ T_{eq}^{4} \sim 1 / r^{2} \] we can calculate the ratio of
equilibrium temperature at perihelion to aphelion using\\
\[ f_T = \frac{T_{eq,per}}{T_{eq,ap}} = \left ( \frac{d_{ap}}{d_{per}} \right )^{1/2} \]
This can the be expressed as a percentage as \(100 \times (f_T - 1)\).
For Earth this is: }

    \begin{Verbatim}[commandchars=\\\{\}]
{\color{incolor}In [{\color{incolor}24}]:} \PY{n}{Trat\PYZus{}earth} \PY{o}{=} \PY{p}{(}\PY{n}{ra\PYZus{}earth} \PY{o}{/} \PY{n}{rp\PYZus{}earth}\PY{p}{)}\PY{o}{*}\PY{o}{*}\PY{l+m+mf}{0.5} 
         
         \PY{k}{print} \PY{p}{(}\PY{n}{Trat\PYZus{}earth} \PY{o}{\PYZhy{}} \PY{l+m+mf}{1.0}\PY{p}{)} \PY{o}{*} \PY{l+m+mi}{100}\PY{p}{,} \PY{l+s+s2}{\PYZdq{}}\PY{l+s+s2}{\PYZpc{}}\PY{l+s+s2}{\PYZdq{}}
\end{Verbatim}

    \begin{Verbatim}[commandchars=\\\{\}]
1.68418033929 \%

    \end{Verbatim}

    \textbf{and for Mars this is:}

    \begin{Verbatim}[commandchars=\\\{\}]
{\color{incolor}In [{\color{incolor}25}]:} \PY{n}{Trat\PYZus{}mars} \PY{o}{=} \PY{p}{(}\PY{n}{ra\PYZus{}mars} \PY{o}{/} \PY{n}{rp\PYZus{}mars}\PY{p}{)}\PY{o}{*}\PY{o}{*}\PY{l+m+mf}{0.5}
         
         \PY{k}{print} \PY{p}{(}\PY{n}{Trat\PYZus{}mars} \PY{o}{\PYZhy{}} \PY{l+m+mf}{1.0}\PY{p}{)} \PY{o}{*} \PY{l+m+mi}{100}\PY{p}{,} \PY{l+s+s2}{\PYZdq{}}\PY{l+s+s2}{\PYZpc{}}\PY{l+s+s2}{\PYZdq{}}
\end{Verbatim}

    \begin{Verbatim}[commandchars=\\\{\}]
9.82006019394 \%

    \end{Verbatim}

    \subsection{Problem 3:}\label{problem-3}

\begin{quote}
Examine the graphs below. Fig. 1 shows calculated gas abundance as
partial pressure (bar) in Jupiter's troposphere as a function of depth
beneath the ammonia cloud layer. Fig. 2 shows the UV flux incident at
Jupiter. Finally, Fig. 3 shows a UV absorption coefficient for selected
gases relevant to Jupiter's atmosphere. In a qualitative way, offer some
general conclusions about how effective the gases in Fig. 3 are in
shielding one another from photolysis in Jupiter's troposphere. Comment
on helium {[}1 pt{]}, methane {[}1 pt{]}, ammonia {[}2 pt{]}, phosphine
{[}1 pt{]} and hydrogen sulfide {[}1 pt{]}. Note that phosphine, PH3, is
present at a much lower abundance in the upper troposphere than ammonia
(NH3), but unlike NH3, phosphine is not condensable. Phosphine
photolyzes to yield free phosphorus, which has many colors (white,
yellow, red, black), which may account for some of Jupiter's colors.
Photolysis of H2S to produce sulfur also will give Jupiter some of its
color. {[}As an example of this kind of discussion: H2 overlies NH3 in
enormous quantities (Fig. 1). However, H2 is incapable of shielding NH3
from photolysis significantly because H2 does not absorb over the wide
wavelength interval from 100 to 235 nm in which the Sun and is very
bright and NH3 absorbs strongly{]}. {[}6 points{]}
\end{quote}

\subsubsection{Answer}\label{answer}

\textbf{Hydrogen and helium absorb short, high energy UV radiation and
both exist in high abundance. They likely help shield all the other
gases from photolysis at these short wavelengths. However, they are
ineffective at shielding longward of 1000 angstroms. Methane is also a
relatively abundant gas in Jupiter's atmosphere and is capable of
shielding other gases from photolysis up to about 1500 angstroms.
Between about 1500-2300 angstroms, ammonia is less protected from
photolysis by the major atmospheric constituents, but does shield
hydrogen sulfide in this range. Hydrogen sulfide appears to be the only
gas that absorbs radiation between about 2300-2800 angstroms, a
wavelength interval over which the Sun increases in brightness by about
an order of magnitude, which seems to result in photolysis and a
subsequent drop in abundance. There seems to be a general trend that,
where there are more UV photons (near UV) there is less shielding, and
the gases that are photochemically active at these wavelengths are
photolyzed deeper in the troposphere, where the optical depth is about
unity for these less abundant gases.}

    \subsection{Problem 4:}\label{problem-4}

\begin{quote}
{[}5 points total{]} At the surface of a planet of Earth mass
(\ldots{}in a galaxy far, far away) about 10\% of the violet (450 nm)
radiation from its sun is scattered during vertical incidence transit
through its atmosphere.
\end{quote}

\begin{quote}
\begin{enumerate}
\def\labelenumi{(\alph{enumi})}
\tightlist
\item
  If the atmosphere has a pressure of 2 bar at the surface and the mean
  molar mass is 38 grams, estimate the mean Rayleigh scattering cross-
  section (i.e.~cross-section (in cm2) per molecule) for the atmospheric
  gas mixture. You can assume that the attenuation is due solely to
  Rayleigh scattering rather than any other form of extinction. {[}3
  pts{]}
\end{enumerate}
\end{quote}

\begin{quote}
\begin{enumerate}
\def\labelenumi{(\alph{enumi})}
\setcounter{enumi}{1}
\tightlist
\item
  Estimate the fraction of incident red (900 nm) light that is scattered
  by passing through the same atmosphere when the sun is 30° above the
  horizon. {[}2 pts{]}
\end{enumerate}
\end{quote}

    \subsubsection{Answer to a)}\label{answer-to-a}

\textbf{If 10\% of the radiation is scattered, then 90\% of the
radiation reaches the surface. Hence, the optical depth at this
wavelength due to Rayleigh scattering can be found using:\\
\[ \frac{I}{I_0} = e^{-\tau} \rightarrow \tau = \ln(10/9) = 1.111 \]\\
At normal incidence the optical depth is related to the cross section
\(\sigma\) and the column density \(N\) via \(\tau = N \sigma\). We can
find \(N\) in terms of the pressure \(P\), gravity \(g\), and mean
molecular weight \(\mu\) as \[ P = N \mu g \] Therefore, the Rayleigh
scattering cross section is\\
\[ \sigma = \frac{\tau \mu g}{P} = \frac{(1.111)(38 \text{ amu})(9.8 \text{ m/s}^2)}{(2 \text{ bar})} = 3.26 \times 10^{-27} \text{ cm}^2 \text{/molecule}\]
}

\subsubsection{Answer to b)}\label{answer-to-b}

\textbf{First we will use our answer from part (a) for the Rayleigh
scattering cross section at 450nm to calculate the cross section at
900nm, using the following scaling relation:
\(\sigma \sim \lambda^{-4}\). Thus,
\[ \sigma_{900} = \sigma_{450} \left ( \frac{900 \text{ nm}}{450 \text{ nm}} \right )^{-4} \\ = ( 3.26 \times 10^{-27} \text{ cm}^2 \text{/molecule} ) \left ( \frac{900 \text{ nm}}{450 \text{ nm}} \right )^{-4} \\ = 2.037 \times 10^{-28} \text{ cm}^2 \text{/molecule}\]\\
Now, we can solve for the ratio of incident flux that reaches the
surface by scaling the column density from above
(\(N \approx 3.23 \times 10^{25} \text{ molecules/cm}^{2}\)) by
\(1/\cos(\theta)\), where \(\theta\) is the solar zenith angle:\\
\[ \frac{I}{I_0} = e^{-\tau} = e^{-N \sigma_{900} / \cos(60^{\circ})} \approx 0.9869\]}

    \begin{Verbatim}[commandchars=\\\{\}]
{\color{incolor}In [{\color{incolor}80}]:} \PY{n}{Irat} \PY{o}{=} \PY{n}{np}\PY{o}{.}\PY{n}{exp}\PY{p}{(} \PY{o}{\PYZhy{}}\PY{p}{(}\PY{l+m+mf}{3.23e25}\PY{p}{)}\PY{o}{*}\PY{p}{(}\PY{l+m+mf}{2.037e\PYZhy{}28}\PY{p}{)} \PY{o}{/} \PY{n}{np}\PY{o}{.}\PY{n}{cos}\PY{p}{(}\PY{l+m+mf}{60.} \PY{o}{*} \PY{n}{np}\PY{o}{.}\PY{n}{pi}\PY{o}{/}\PY{l+m+mf}{180.}\PY{p}{)} \PY{p}{)}
         \PY{k}{print} \PY{n}{Irat}
\end{Verbatim}

    \begin{Verbatim}[commandchars=\\\{\}]
0.986927181381

    \end{Verbatim}

    \textbf{Only 1.3\% of the light at 900nm is Rayleigh scattered near
sunset. This is why the sky appears red in the direction of the sun at
sunset.}

    \subsection{Problem 5:}\label{problem-5}

\begin{quote}
{[}16 points total{]} Astronomers discover a large asteroid on a
collision course with the Earth. It is expected to hit somewhere in the
Middle East. Sadly, it is too late to stop the asteroid. But it can be
deflected so that it lands in an unpopulated area. Government A argues
that the asteroid should be deflected into the desert of Iraq. However,
Government B argues that that the asteroid should be deflected into the
Indian Ocean.
\end{quote}

\begin{quote}
The question is: which is worse for the inhabitants of the Earth: a land
impact or an ocean splash?
\end{quote}

\begin{quote}
For climate risk assessment, you have the following information: * In
the event of a desert impact, dust lifted into the atmosphere will
decrease the planetary albedo for reflected sunlight from the top of the
atmosphere from 0.3 to 0.25, while the absorption of solar radiation in
the atmosphere will increase from 20\% to 50\% as a percentage of the
incoming solar flux. * If the asteroid lands in water, the amount of
water vapor introduced into the atmosphere will make the lower
atmosphere entirely opaque to thermal infrared radiation for several
weeks.
\end{quote}

\begin{quote}
\begin{enumerate}
\def\labelenumi{(\alph{enumi})}
\tightlist
\item
  The media dubs one scenario analogous to a ``nuclear winter'' and the
  other a ``hothouse Earth''. Before calculating anything, which of the
  scenarios do you think will be analogous to a ``nuclear winter'' and
  which would be the ``hothouse''. {[}2 pts{]}
\end{enumerate}
\end{quote}

\subsubsection{Answer to a)}\label{answer-to-a}

\textbf{I would think that the desert impact would be the ``nuclear
winter'' because the atmosphere will absorb more radiation and reradiate
half of it back to space. The ``hothouse Earth'' would be the water
impact beacuse if the atmosphere becomes entirely opaque to thermal IR
radiation then the Earth will be unable to cool and will lead to a rise
in surface temperature. }

\begin{quote}
\begin{enumerate}
\def\labelenumi{(\alph{enumi})}
\setcounter{enumi}{1}
\tightlist
\item
  Consider a simplified model for the impact on land where the
  atmosphere consists of a single, slab layer above the surface at
  temperature T1. Assume the ground is at temperature Tg. Before the
  impact, the time- and space-averaged solar radiation on the planet is
  reflected according to albedo, and the atmosphere absorbs 20\% of the
  solar radiation (total, not minus the reflected amount) before it
  reaches the ground. Assume the surface emits like a blackbody and
  assume the atmospheric layer emits like a blackbody. Take the solar
  constant S = 1360 W m-2. The Stefan- Boltzmann constant = σ = 5.67×
  10-8 W m-2 K-4.
\end{enumerate}
\end{quote}

\begin{quote}
\begin{enumerate}
\def\labelenumi{(\roman{enumi})}
\tightlist
\item
  Evaluate the temperature of the ground before the impact. (Hint: Draw
  a diagram and consider a balance of fluxes in and out of each of the
  atmosphere layer, the ground, and top of atmosphere; remember that the
  atmospheric layer emits downward and upward. Apply the
  Stefan-Boltzmann law, in particular). {[}4 pts{]}\\
\item
  Evaluate the temperature of the ground after impact. {[}4 pts{]}\\
\item
  Explain qualitatively why, in a more complex calculation, the ground
  temperature is likely to depend on the vertical position of the dust
  in the atmosphere (e.g., troposphere versus stratosphere). {[}4 pts{]}
\end{enumerate}
\end{quote}

\subsubsection{Answer to b)}\label{answer-to-b}

\textbf{To maintain flux balance, we get the following two equations,
one for each radiating layer/surface. In the atmospheric layer at
temperature \(T_1\),\\
\[ \frac{\alpha S_{\oplus}}{4} = 2 \sigma T_1^4 \] where \(\alpha\) is
the raction of flux absorbed by the atmosphere layer, and the \(2\) on
the RHS comes from the fact the radiation is emitted upwards and
downwards. For the surface, at temperature \(T_g\), we have
\[ \frac{S_{\oplus}}{4}(1 - A - \alpha) + \sigma T_1^4 = \sigma T_g^4 \]
Solving this system of equations for the ground temperature gives:
\[ T_g = \left [\frac{S_{\oplus}}{4 \sigma} (1 - A - \alpha/2) \right ]^{1/4} \]
Plugging in the numbers\ldots{}}

    \begin{Verbatim}[commandchars=\\\{\}]
{\color{incolor}In [{\color{incolor}82}]:} \PY{n}{A} \PY{o}{=} \PY{l+m+mf}{0.3}          \PY{c+c1}{\PYZsh{} Albedo of Earth initially      }
         \PY{n}{S\PYZus{}e} \PY{o}{=} \PY{l+m+mf}{1360.}      \PY{c+c1}{\PYZsh{} Solar Constant  }
         \PY{n}{f} \PY{o}{=} \PY{l+m+mf}{0.2}          \PY{c+c1}{\PYZsh{} Fraction of flux absorbed by atmosphere initially}
         \PY{n}{sigma} \PY{o}{=} \PY{l+m+mf}{5.67e\PYZhy{}8}  \PY{c+c1}{\PYZsh{} Stefan\PYZhy{}Boltzmann constant}
         
         \PY{n}{T\PYZus{}g} \PY{o}{=} \PY{p}{(}\PY{p}{(}\PY{n}{S\PYZus{}e}\PY{p}{)} \PY{o}{/} \PY{p}{(}\PY{l+m+mf}{4.} \PY{o}{*} \PY{n}{sigma}\PY{p}{)} \PY{o}{*} \PY{p}{(}\PY{l+m+mf}{1.} \PY{o}{\PYZhy{}} \PY{n}{A} \PY{o}{\PYZhy{}} \PY{n}{f}\PY{o}{/}\PY{l+m+mi}{2}\PY{p}{)}\PY{p}{)}\PY{o}{*}\PY{o}{*}\PY{p}{(}\PY{l+m+mf}{0.25}\PY{p}{)}
         
         \PY{k}{print} \PY{n}{T\PYZus{}g}\PY{p}{,} \PY{l+s+s2}{\PYZdq{}}\PY{l+s+s2}{K}\PY{l+s+s2}{\PYZdq{}}
\end{Verbatim}

    \begin{Verbatim}[commandchars=\\\{\}]
244.912965608 K

    \end{Verbatim}

    \textbf{(i) We find an initial surface temperature of 244.9K. A bit
cold, but we'll work with it. After the land impact\ldots{}}

    \begin{Verbatim}[commandchars=\\\{\}]
{\color{incolor}In [{\color{incolor}83}]:} \PY{n}{A} \PY{o}{=} \PY{l+m+mf}{0.25}         \PY{c+c1}{\PYZsh{} Albedo of Earth after impact      }
         \PY{n}{S\PYZus{}e} \PY{o}{=} \PY{l+m+mf}{1360.}      \PY{c+c1}{\PYZsh{} Solar Constant  }
         \PY{n}{f} \PY{o}{=} \PY{l+m+mf}{0.5}          \PY{c+c1}{\PYZsh{} Fraction of flux absorbed by atmosphere after impact}
         \PY{n}{sigma} \PY{o}{=} \PY{l+m+mf}{5.67e\PYZhy{}8}  \PY{c+c1}{\PYZsh{} Stefan\PYZhy{}Boltzmann constant}
         
         \PY{n}{T\PYZus{}g} \PY{o}{=} \PY{p}{(}\PY{p}{(}\PY{n}{S\PYZus{}e}\PY{p}{)} \PY{o}{/} \PY{p}{(}\PY{l+m+mf}{4.} \PY{o}{*} \PY{n}{sigma}\PY{p}{)} \PY{o}{*} \PY{p}{(}\PY{l+m+mf}{1.} \PY{o}{\PYZhy{}} \PY{n}{A} \PY{o}{\PYZhy{}} \PY{n}{f}\PY{o}{/}\PY{l+m+mi}{2}\PY{p}{)}\PY{p}{)}\PY{o}{*}\PY{o}{*}\PY{p}{(}\PY{l+m+mf}{0.25}\PY{p}{)}
         
         \PY{k}{print} \PY{n}{T\PYZus{}g}\PY{p}{,} \PY{l+s+s2}{\PYZdq{}}\PY{l+s+s2}{K}\PY{l+s+s2}{\PYZdq{}}
\end{Verbatim}

    \begin{Verbatim}[commandchars=\\\{\}]
234.000327707 K

    \end{Verbatim}

    \textbf{(ii) We find that the temperature of the Earth after the impact
has dropped to 234K.}

\textbf{(iii) The vertical location of the dust in the atmosphere will
influence this result for a few reasons. First, if the dust were in the
troposphere then more of the atmosphere above it would be able to absorb
radiation. Also, since convection occurs in the troposphere, the amount
of time the dust remains climatically active in the atmosphere will
probably be much longer if it reaches the stratosphere, rather than
settling in the troposphere.}

\begin{quote}
\begin{enumerate}
\def\labelenumi{(\alph{enumi})}
\setcounter{enumi}{2}
\tightlist
\item
  For the ocean impact, consider a modified model. Add a second layer to
  the atmosphere at temperature T2 beneath the layer at temperature T1.
  The lower layer is opaque to thermal infrared radiation from the
  ground. Assume that the absorption of total solar radiation is equally
  divided between the two atmospheric layers. What is temperature of the
  ground after the impact into the Indian Ocean, using this simple
  model? {[}4 pts{]}
\end{enumerate}
\end{quote}

\subsubsection{Answer to c)}\label{answer-to-c}

\textbf{Similar to part (a), enforcing flux balance now gives three
equations: \[ \frac{\alpha S_{\oplus}}{8} = 2 \sigma T_1^4 \\
\frac{\alpha S_{\oplus}}{8} + \sigma T_1^4 + \sigma T_g^4 = 2 \sigma T_2^4 \\ 
\frac{\alpha S_{\oplus}}{4}(1 - A - \alpha) + \sigma T_2^4 = \sigma T_g^4\]
where layer 2 of the atmosphere is optically thick to thermal radiation
from the surface and from layer 1. Solving this system for \(T_g\)
requires lots of algebra and gives:
\[ T_g = \left [ \frac{S_{\oplus}}{2 \sigma} \left ( 1 - A - \frac{5}{8} \alpha \right ) \right ]^{1/4} \]}

    \begin{Verbatim}[commandchars=\\\{\}]
{\color{incolor}In [{\color{incolor}84}]:} \PY{n}{A} \PY{o}{=} \PY{l+m+mf}{0.3}          \PY{c+c1}{\PYZsh{} Albedo of Earth    }
         \PY{n}{S\PYZus{}e} \PY{o}{=} \PY{l+m+mf}{1360.}      \PY{c+c1}{\PYZsh{} Solar Constant  }
         \PY{n}{f} \PY{o}{=} \PY{l+m+mf}{0.2}          \PY{c+c1}{\PYZsh{} Fraction of flux absorbed by atmosphere after impact}
         \PY{n}{sigma} \PY{o}{=} \PY{l+m+mf}{5.67e\PYZhy{}8}  \PY{c+c1}{\PYZsh{} Stefan\PYZhy{}Boltzmann constant}
         
         \PY{n}{T\PYZus{}g} \PY{o}{=} \PY{p}{(}\PY{p}{(}\PY{n}{S\PYZus{}e}\PY{p}{)} \PY{o}{/} \PY{p}{(}\PY{l+m+mf}{2.} \PY{o}{*} \PY{n}{sigma}\PY{p}{)} \PY{o}{*} \PY{p}{(}\PY{l+m+mf}{1.} \PY{o}{\PYZhy{}} \PY{n}{A} \PY{o}{\PYZhy{}} \PY{p}{(}\PY{l+m+mf}{5.}\PY{o}{/}\PY{l+m+mf}{8.}\PY{p}{)}\PY{o}{*}\PY{n}{f}\PY{p}{)}\PY{p}{)}\PY{o}{*}\PY{o}{*}\PY{p}{(}\PY{l+m+mf}{0.25}\PY{p}{)}
         
         \PY{k}{print} \PY{n}{T\PYZus{}g}\PY{p}{,} \PY{l+s+s2}{\PYZdq{}}\PY{l+s+s2}{K}\PY{l+s+s2}{\PYZdq{}}
\end{Verbatim}

    \begin{Verbatim}[commandchars=\\\{\}]
288.169773147 K

    \end{Verbatim}

    \textbf{After the water impact, we find the surface temperature to be
288K.}

    \subsection{Problem 6:}\label{problem-6}

\begin{quote}
{[}4 points total{]} Jupiter's atmosphere receives energy from the Sun
and from Jupiter's interior. In this question, you may find the
following information useful: Jupiter's Bond albedo = 0.343, the solar
constant at Earth is 1360 W m-2. The mean Earth-Sun distance is
1.496×1011 m; the mean Jupiter-Sun distance is 7.78×1011 m; Jupiter's
radius = 71,492 km. Stefan-Boltzmann constant = 5.67 ×10−8 W m−2 K−4.
\end{quote}

\begin{quote}
\begin{enumerate}
\def\labelenumi{(\roman{enumi})}
\tightlist
\item
  What is the equilibrium temperature of Jupiter based on the input of
  solar energy only? {[}2 pts{]}
\end{enumerate}
\end{quote}

\begin{quote}
\begin{enumerate}
\def\labelenumi{(\roman{enumi})}
\setcounter{enumi}{1}
\tightlist
\item
  Spacecraft have measured the effective temperature of Jupiter as 124.4
  K. The difference between the measured and the calculated value of the
  temperature is thought to be due to energy from the interior of the
  planet. How large a flux (W m-2) is required to account for the
  difference? How much energy (W) is emanating from the interior of the
  Jupiter? (Comparisons: 15 TW energy use by human civilization or 45 TW
  emanating from the Earth's interior).
\end{enumerate}
\end{quote}

\subsubsection{Answer to a)}\label{answer-to-a}

\textbf{We'll first scale the solar constant at Earth to Jupiter,\\
\[ S_J = S_{\oplus} \left ( \frac{d_{\oplus}}{d_J} \right )^{2} \]}

    \begin{Verbatim}[commandchars=\\\{\}]
{\color{incolor}In [{\color{incolor}50}]:} \PY{n}{S\PYZus{}J} \PY{o}{=} \PY{l+m+mf}{1360.} \PY{o}{*} \PY{p}{(}\PY{l+m+mf}{1.496e11} \PY{o}{/} \PY{l+m+mf}{7.78e11}\PY{p}{)}\PY{o}{*}\PY{o}{*}\PY{l+m+mi}{2}\PY{p}{;} \PY{k}{print} \PY{n}{S\PYZus{}J}\PY{p}{,} \PY{l+s+s2}{\PYZdq{}}\PY{l+s+s2}{W/m\PYZca{}2}\PY{l+s+s2}{\PYZdq{}}
\end{Verbatim}

    \begin{Verbatim}[commandchars=\\\{\}]
50.2855148988 W/m\^{}2

    \end{Verbatim}

    \textbf{Now we can solve for the equilibrium temperature of Jupiter
based on the solar input only:\\
\[ T_{eq} = \left [ \frac{(1-A)S_J}{4 \sigma}\right ]^{1/4} \]}

    \begin{Verbatim}[commandchars=\\\{\}]
{\color{incolor}In [{\color{incolor}51}]:} \PY{n}{A} \PY{o}{=} \PY{l+m+mf}{0.343}        \PY{c+c1}{\PYZsh{} Albedo of Jupiter      }
         \PY{n}{sigma} \PY{o}{=} \PY{l+m+mf}{5.67e\PYZhy{}8}  \PY{c+c1}{\PYZsh{} Stefan\PYZhy{}Boltzmann constant}
         
         \PY{n}{T\PYZus{}eq1} \PY{o}{=} \PY{p}{(}\PY{p}{(}\PY{p}{(}\PY{l+m+mf}{1.}\PY{o}{\PYZhy{}}\PY{n}{A}\PY{p}{)}\PY{o}{*}\PY{n}{S\PYZus{}J}\PY{p}{)}\PY{o}{/}\PY{p}{(}\PY{l+m+mf}{4.}\PY{o}{*}\PY{n}{sigma}\PY{p}{)}\PY{p}{)}\PY{o}{*}\PY{o}{*}\PY{p}{(}\PY{l+m+mf}{0.25}\PY{p}{)}
         
         \PY{k}{print} \PY{n}{T\PYZus{}eq1}\PY{p}{,} \PY{l+s+s2}{\PYZdq{}}\PY{l+s+s2}{K}\PY{l+s+s2}{\PYZdq{}}
\end{Verbatim}

    \begin{Verbatim}[commandchars=\\\{\}]
109.860432695 K

    \end{Verbatim}

    \textbf{We find \(T_{eq} \approx 110\) K for Jupiter assuming only solar
heating.}

\subsubsection{Answer to b)}\label{answer-to-b}

\textbf{If the measured equilibrium temperature of Jupiter is:}

    \begin{Verbatim}[commandchars=\\\{\}]
{\color{incolor}In [{\color{incolor}55}]:} \PY{n}{T\PYZus{}J} \PY{o}{=} \PY{l+m+mf}{124.4} 
         \PY{k}{print} \PY{n}{T\PYZus{}J}\PY{p}{,} \PY{l+s+s2}{\PYZdq{}}\PY{l+s+s2}{K}\PY{l+s+s2}{\PYZdq{}}
\end{Verbatim}

    \begin{Verbatim}[commandchars=\\\{\}]
124.4 K

    \end{Verbatim}

    \textbf{There must be internal heating, which can be accounted for via
an additional flux term, \(F_i\),\\
\[ \sigma T_{eq}^4 = \frac{(1-A)S_J}{4} + F_i\] which we will solve for
\[F_i = \sigma T_{eq}^4 - \frac{(1-A)S_J}{4}\]}

    \begin{Verbatim}[commandchars=\\\{\}]
{\color{incolor}In [{\color{incolor}56}]:} \PY{n}{F\PYZus{}i} \PY{o}{=} \PY{n}{sigma} \PY{o}{*} \PY{n}{T\PYZus{}J}\PY{o}{*}\PY{o}{*}\PY{l+m+mi}{4} \PY{o}{\PYZhy{}} \PY{p}{(}\PY{p}{(}\PY{p}{(}\PY{l+m+mf}{1.}\PY{o}{\PYZhy{}}\PY{n}{A}\PY{p}{)}\PY{o}{*}\PY{n}{S\PYZus{}J}\PY{p}{)}\PY{o}{/}\PY{p}{(}\PY{l+m+mf}{4.}\PY{p}{)}\PY{p}{)}
         \PY{k}{print} \PY{n}{F\PYZus{}i}\PY{p}{,} \PY{l+s+s2}{\PYZdq{}}\PY{l+s+s2}{W/m\PYZca{}2}\PY{l+s+s2}{\PYZdq{}}
\end{Verbatim}

    \begin{Verbatim}[commandchars=\\\{\}]
5.31950387412 W/m\^{}2

    \end{Verbatim}

    \textbf{A flux of roughly \(5.3\) W/m\(^2\) is required to explain
Jupiter's equilibrium temperature. Multipyling this by Jupiter's surface
area (\(4 \pi R_J^2\)) will give us the total energy emanating from
Jupiter's interior: }

    \begin{Verbatim}[commandchars=\\\{\}]
{\color{incolor}In [{\color{incolor}57}]:} \PY{n}{R\PYZus{}J} \PY{o}{=} \PY{l+m+mf}{71492000.}    \PY{c+c1}{\PYZsh{} Radius of Jupiter in meters}
         \PY{n}{E\PYZus{}J} \PY{o}{=} \PY{n}{F\PYZus{}i} \PY{o}{*} \PY{l+m+mf}{4.} \PY{o}{*} \PY{n}{np}\PY{o}{.}\PY{n}{pi} \PY{o}{*} \PY{n}{R\PYZus{}J}\PY{o}{*}\PY{o}{*}\PY{l+m+mi}{2}
         \PY{k}{print} \PY{n}{E\PYZus{}J}\PY{p}{,} \PY{l+s+s2}{\PYZdq{}}\PY{l+s+s2}{Watts}\PY{l+s+s2}{\PYZdq{}}
\end{Verbatim}

    \begin{Verbatim}[commandchars=\\\{\}]
3.41661377024e+17 Watts

    \end{Verbatim}

    \textbf{This is a massive 341 petawatts! Which is comparable to the
total power intercepted by the Earth from the Sun. Even though the flux
is much lower than the solar constant, Jupiter's large radius makes the
total power radiated from the interior so immense.}

    \subsection{Problem 7:}\label{problem-7}

\begin{quote}
The figures below are coincident measurements of IR emission of the
terrestrial cloud-free atmosphere (a) 20 km looking downward over a
polar ice sheet (b) at the surface looking upward. Answer the following.
In each case explain how you deduce the answer with reference to the
part of the spectrum or feature that you're looking at:
\end{quote}

\begin{quote}
\begin{enumerate}
\def\labelenumi{(\alph{enumi})}
\tightlist
\item
  What is the approximate temperature of the surface of the ice sheet?
  {[}3 pts{]}
\end{enumerate}
\end{quote}

\textbf{The surface of the ice sheet is approximately 267K, which I
deduce from the continuum flux in panel (a).}

\begin{quote}
\begin{enumerate}
\def\labelenumi{(\alph{enumi})}
\setcounter{enumi}{1}
\tightlist
\item
  What is the approximate temperature of the near-surface air? {[}3
  pts{]}
\end{enumerate}
\end{quote}

\textbf{The temperature of the near-surface air is approximately 270K,
which I deduce from the core of the 15 micron CO2 band in panel (b). As
stated in the reading, at the surface of the Earth the core of the band
is approximately 95\% absorbed in just one meter! This makes it very
sensitive to the temperature near the detector.}

\begin{quote}
\begin{enumerate}
\def\labelenumi{(\alph{enumi})}
\setcounter{enumi}{2}
\tightlist
\item
  What is the approximate temperature of the air at the aircraft's
  altitude of 20 km? {[}3 pts{]}
\end{enumerate}
\end{quote}

\textbf{Similarly, we can use the core of the 15 micron CO2 band in
panel (a) to estimate the temperature at the aircraft height of 20 km to
be approximately 230K.}

\begin{quote}
\begin{enumerate}
\def\labelenumi{(\alph{enumi})}
\setcounter{enumi}{3}
\tightlist
\item
  Which band is responsible for the feature between 9 and 10 microns?
  {[}1pt{]}
\end{enumerate}
\end{quote}

\textbf{Ozone absorbs between 9 and 10 microns.}

\begin{quote}
\begin{enumerate}
\def\labelenumi{(\alph{enumi})}
\setcounter{enumi}{4}
\tightlist
\item
  Is there any evidence for an inversion in the near-surface temperature
  profile or not? {[}2 pts{]}
\end{enumerate}
\end{quote}

\textbf{There may be evidence of a near-surface inversion in the wings
of the 15 micron CO2 band in panel (b). It looks like the wings have a
slightly higher radiance than much of the rest of the band (not the core
tho), but we know that the absorption cross-section is lower here, which
implies that the light is emitted slightly farther away (higher up), yet
it is warmer there. This may imply that it is slightly warmer just above
the surface, where there is perhaps less cooling of air from the
ice\ldots{}}

\subsection{Problem 8:}\label{problem-8}

\begin{quote}
\begin{enumerate}
\def\labelenumi{\arabic{enumi})}
\setcounter{enumi}{7}
\tightlist
\item
  {[}4 points total{]}. A scientist wants to measure the
  top-of-atmosphere solar intensity. She uses a ground-based radiometer
  operating at λ = 0.45 μm to measure the solar intensity at the ground,
  Iλ(0). For a solar zenith angle θ = 30°, Iλ(0) = 1.74×107 W m-2 μm-1
  sr-1. For θ = 60°, Iλ(0) = 1.14×107 W m-2 μm-1 sr-1. From this
  information, determine the top-of-the-atmosphere solar intensity Sλ
  (in W m-2 μm-1 sr-1 ) and the atmospheric optical thickness, τλ.
\end{enumerate}
\end{quote}

\subsubsection{Answer}\label{answer}

\textbf{Let's denote her first measurement \(I_1\) at \(\theta_1=30\)
degrees, and her second measurement \(I_2\) at \(\theta_2=60\) degrees.
We can use the Beer-Lambert Law for each measurement to set up a system
of 2 equations and 2 unknowns:
\[ I_1 = I_0 e^{-\tau / \cos(\theta_1)} \\ I_2 = I_0 e^{-\tau / \cos(\theta_2)}\]
We can now solve for the common optical depth by dividing one equation
by the other and solving for \(\tau\). Then we get:
\[ \tau = \frac{\ln(I_1/I_2)}{\sec(\theta_2) - \sec(\theta_1)} \]}

    \begin{Verbatim}[commandchars=\\\{\}]
{\color{incolor}In [{\color{incolor}61}]:} \PY{c+c1}{\PYZsh{} The measured intensities}
         \PY{n}{I1} \PY{o}{=} \PY{l+m+mf}{1.74e7}
         \PY{n}{I2} \PY{o}{=} \PY{l+m+mf}{1.14e7}
         
         \PY{c+c1}{\PYZsh{} The measurement solar zenith angles}
         \PY{n}{theta1} \PY{o}{=} \PY{l+m+mf}{30.} \PY{o}{*} \PY{p}{(}\PY{n}{np}\PY{o}{.}\PY{n}{pi} \PY{o}{/} \PY{l+m+mf}{180.}\PY{p}{)}
         \PY{n}{theta2} \PY{o}{=} \PY{l+m+mf}{60.} \PY{o}{*} \PY{p}{(}\PY{n}{np}\PY{o}{.}\PY{n}{pi} \PY{o}{/} \PY{l+m+mf}{180.}\PY{p}{)}
         
         \PY{c+c1}{\PYZsh{} Calculate the optical depth using the above equation}
         \PY{n}{tau} \PY{o}{=} \PY{p}{(}\PY{n}{np}\PY{o}{.}\PY{n}{log}\PY{p}{(}\PY{n}{I1}\PY{o}{/}\PY{n}{I2}\PY{p}{)}\PY{p}{)} \PY{o}{/} \PY{p}{(}\PY{l+m+mf}{1.}\PY{o}{/}\PY{n}{np}\PY{o}{.}\PY{n}{cos}\PY{p}{(}\PY{n}{theta2}\PY{p}{)} \PY{o}{\PYZhy{}} \PY{l+m+mi}{1}\PY{o}{/}\PY{n}{np}\PY{o}{.}\PY{n}{cos}\PY{p}{(}\PY{n}{theta1}\PY{p}{)}\PY{p}{)}
         
         \PY{k}{print} \PY{n}{tau}
\end{Verbatim}

    \begin{Verbatim}[commandchars=\\\{\}]
0.500245025602

    \end{Verbatim}

    \textbf{We find the optical thickness to be \(\tau = 0.5\). Now we can
solve of the intensity incident at top of atmosphere by solving either
of the above equations for \(I_0\):\\
\[ I_0 =  \frac{I_1}{e^{-\tau / \cos(\theta_1)}}\]}

    \begin{Verbatim}[commandchars=\\\{\}]
{\color{incolor}In [{\color{incolor}66}]:} \PY{n}{I0} \PY{o}{=} \PY{n}{I1} \PY{o}{/} \PY{p}{(}\PY{n}{np}\PY{o}{.}\PY{n}{exp}\PY{p}{(}\PY{o}{\PYZhy{}}\PY{n}{tau} \PY{o}{/} \PY{n}{np}\PY{o}{.}\PY{n}{cos}\PY{p}{(}\PY{n}{theta1}\PY{p}{)}\PY{p}{)}\PY{p}{)}
         
         \PY{k}{print} \PY{n}{I0} \PY{o}{/} \PY{l+m+mf}{1e7}\PY{p}{,} \PY{l+s+s2}{\PYZdq{}}\PY{l+s+s2}{e7}\PY{l+s+s2}{\PYZdq{}}
\end{Verbatim}

    \begin{Verbatim}[commandchars=\\\{\}]
3.1003602475 e7

    \end{Verbatim}

    \textbf{We find \(I_0 = 3.1 \times 10^7\) W/m\^{}2/um/sr.}

\subsection{Problem 9:}\label{problem-9}

\begin{quote}
\begin{enumerate}
\def\labelenumi{\arabic{enumi})}
\setcounter{enumi}{8}
\tightlist
\item
  In another galaxy far away, an Earth-sized exoplanet in an Earth-sized
  orbit has an albedo of 0.3, which is the same as the Earth's. But this
  exoplanet exists around a hotter star with a maximum emission per unit
  wavelength of 0.4 μm instead of 0.502 μm for our Sun. What is the
  difference in effective temperature for this exoplanet compared to the
  Earth's 255 K? (For the sake of this question, we will ignore an
  astronomical nuance that a hotter star will tend to have greater
  radius that the Sun and so more luminosity just due to size. In this
  question, we assume that the star has the same size.). {[}3 points{]}.
\end{enumerate}
\end{quote}

\textbf{Using Wien's Law:
\[ T_{eff} = \frac{2897 \text{ $\mu$m K}}{\lambda_{max}} \]}

    \begin{Verbatim}[commandchars=\\\{\}]
{\color{incolor}In [{\color{incolor}68}]:} \PY{n}{b} \PY{o}{=} \PY{l+m+mf}{2897.}
         
         \PY{n}{T\PYZus{}sun} \PY{o}{=} \PY{n}{b} \PY{o}{/} \PY{l+m+mf}{0.502}\PY{p}{;} \PY{k}{print} \PY{n}{T\PYZus{}sun}\PY{p}{,} \PY{l+s+s2}{\PYZdq{}}\PY{l+s+s2}{K}\PY{l+s+s2}{\PYZdq{}}
         
         \PY{n}{T\PYZus{}star} \PY{o}{=} \PY{n}{b} \PY{o}{/} \PY{l+m+mf}{0.4}\PY{p}{;} \PY{k}{print} \PY{n}{T\PYZus{}star}\PY{p}{,} \PY{l+s+s2}{\PYZdq{}}\PY{l+s+s2}{K}\PY{l+s+s2}{\PYZdq{}}
\end{Verbatim}

    \begin{Verbatim}[commandchars=\\\{\}]
5770.91633466 K
7242.5 K

    \end{Verbatim}

    \textbf{All things being equal, the stellar effective temperature scales
linearly with the planetary equilibrium temperature. Therefore, the
temperature of the planet will be
\[ T_p = 255 \text{ K} \left ( \frac{7242 \text{ K}}{5770 \text{ K}}\right ) \approx 320 \text{ K}\]}

    \begin{Verbatim}[commandchars=\\\{\}]
{\color{incolor}In [{\color{incolor}71}]:} \PY{k}{print} \PY{l+m+mi}{255} \PY{o}{*} \PY{p}{(}\PY{n}{T\PYZus{}star}\PY{o}{/}\PY{n}{T\PYZus{}sun}\PY{p}{)}\PY{p}{,} \PY{l+s+s2}{\PYZdq{}}\PY{l+s+s2}{K}\PY{l+s+s2}{\PYZdq{}}
\end{Verbatim}

    \begin{Verbatim}[commandchars=\\\{\}]
320.025 K

    \end{Verbatim}

    \begin{Verbatim}[commandchars=\\\{\}]
{\color{incolor}In [{\color{incolor} }]:} 
\end{Verbatim}


    % Add a bibliography block to the postdoc
    
    
    
    \end{document}
